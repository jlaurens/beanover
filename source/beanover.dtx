% \iffalse meta-comment
% !TEX program  = LuaLaTeX
%
%% File: beanover.dtx
%
% Run this file twice with LuaLaTeX
%
% Copyright (C) 2022 Jérôme Laurens
%
% This file may be distributed and/or modified under the
% conditions of the LaTeX Project Public License, either
% version 1.3c of this license or (at your option) any later
% version. The latest version of this license is in:
%
%     http://www.latex-project.org/lppl.txt
%
% and version 1.3c or later is part of all distributions of
% LaTeX version 2020/01/01 or later.
%
%<*internal>
\iffalse
%</internal>
%<*readme>
----------------------------------------------------------------
beanover --- beamer named overlay ranges
E-mail: jerome.laurens@u-bourgogne.fr
Released under the LaTeX Project Public License v1.3c or later
See https://www.latex-project.org/lppl.txt
----------------------------------------------------------------

This package allows the management of multiple time lines in |beamer| documents.
Time lines are very handy both during edition and to manage complex and variable overlay specifications.
%</readme>
%<*internal>
\fi
\def\nameofplainTeX{plain}
\ifx\fmtname\nameofplainTeX\else
  \expandafter\begingroup
\fi
%</internal>
%<*install>
\input docstrip.tex
\keepsilent
\askforoverwritefalse
\preamble
\endpreamble
\postamble

beanover --- beamer named overlay ranges

Copyright (C) 2022 by Jérôme LAURENS <jerome.laurens@u-bourgogne.fr>

This work may be distributed and/or modified under the
conditions of the LaTeX Project Public License (LPPL), either
version 1.3c of this license or (at your option) any later
version.  The latest version of this license is in the file:

https://www.latex-project.org/lppl.txt

This work is "maintained" (as per LPPL maintenance status) by
Jérôme LAURENS.

This work consists of the file  beanover.dtx
and the derived files           beanover.ins,
                                beanover.pdf and
                                beanover.sty.

\endpostamble
\usedir{tex/latex/beanover}
\generate{
  \file{\jobname.sty}{\from{\jobname.dtx}{package}}
}
%</install>
%<install>\endbatchfile
%<*internal>
\usedir{source/latex/beanover}
\generate{
  \file{\jobname.ins}{\from{\jobname.dtx}{install}}
}
\nopreamble\nopostamble
\usedir{doc/latex/beanover}
\generate{
  \file{README.txt}{\from{\jobname.dtx}{readme}}
}
\ifx\fmtname\nameofplainTeX
  \expandafter\endbatchfile
\else
  \expandafter\endgroup
\fi
%</internal>
%<*driver>
% !TeX encoding = UTF-8
% !TeX program = lualatex
% ! TeX proot = ...
\ProvideDocumentCommand\directlua{m}{\wlog{Run with LuaLaTeX}}
\directlua {
  local cmd = 'pdftex "\jobname.dtx"'
  local f = assert(io.popen(cmd, 'r')) 
  local s = assert(f:read('*a')) 
  f:close() 
}
\documentclass{l3doc}
\RequirePackage{hyperref}
\RequirePackage{multirow}
\RequirePackage{tcolorbox}
\RequirePackage{unicode}
\RequirePackage{unicode-math}
\RequirePackage{emoji}
\tcbuselibrary{minted}
\RequirePackage{beanover}
\EnableCrossrefs
\CodelineIndex
\RecordChanges
\title{\pkg{beamer} named overlay ranges with \pkg{beanover}}
\author{Jérôme Laurens}
\GetFileInfo{beanover.sty}
\date{\fileversion \qquad \filedate}
\begin{document}
\maketitle
\changes{v1.0}{2009/10/06}{First public release}
\begin{abstract}
This package allows the management of multiple slide lists in |beamer| documents.
Slide ranges are very handy both during edition and to manage complex and variable overlay specifications.
\end{abstract}
%
\tableofcontents
%
\begin{documentation}
%
\section{Minimal example}
%
The document below is a contrived example to show how the |beamer|
overlay specifications have been extended.

\begin{tcblisting} {
  listing only,
  listing file = example1.tex,
  minted options={
    fontsize=\small,
    breaklines,
    linenos,
    numbersep=0.5\baselineskip,
  },
  minted style = autumn,
}
\documentclass {beamer}
\RequirePackage {beanover}
\begin{document}
\begin{frame} [
    beanover = {
      A = 1:2,
      B = A.next:3,
      C = B.next,
    }
  ]
  {\Large Frame \insertframenumber}
  {\Large Slide \insertslidenumber}
\visible<?(A.1)> {Only on slide 1}\\
\visible<?(B.1)-?(B.last)> {Only on slide 3 to 5}\\
\visible<?(C.1)> {Only on slide 6}\\
\visible<?(A.2)> {Only on slide 2}\\
\visible<?(B.2)-?(B.last)> {Only on slide 4 to 5}\\
\visible<?(C.2)> {Only on slide 7}\\
\visible<?(A.3)-> {From slide 3}\\
\visible<?(B.3)-?(B.last)> {Only on slide 5}\\
\visible<?(C.3)> {Only on slide 8}\\
\end{frame}
\end{document}
\end{tcblisting}
%
On line 5, we use the |beanover| key to declare named slide ranges.
On line 6, we declare a slide range named `A', starting at slide 1 and with length 2.
On line 13, the new overlay specification \texttt{?(A.1)} stands for 1,
on line 16, \texttt{?(A.2)} stands for 2
and on line 19, \texttt{?(A.3)} stands for 3.
%
On line 7, we declare a second slide range named `B',
starting after the 2 slides of `A' namely 3.
Its length is 3 meaning that its last slide number is 5,
thus each \texttt{?(B.last)} is replaced by 5.
The next slide number after slide range `B' is 6
which is also the start of the third slide range
due to line 8.
\section{Named slide lists}
\subsection{Presentation}
Within a \pkg{beamer} frame, there are different slides that appear in turn.
The main slide list is a range on integers coversing all the slide numbers, from one to the total amount of slides.
In general, a slide list is a range of positive integers identified by a unique name.
The main practical interest is that such lists may be defined relative to one another, we can even have lists of slide ranges.
Finally, we can use these lists to specify \pkg{beamer} overlay specifications.
%
\subsection{Defining named slide ranges}
In order to define named slide ranges, we can either use the |\Beanover| command below inside a \pkg{beamer} frame environment, or use the |beanover| option of this environment.
The value of the |beanover| option is exactly the argument of the |\Beanover| command.
When used, the |\Beanover| command is executed on each frame, whereas the option is executed only once but is more verbose.
%
\begin{function}{\Beanover}
  \begin{syntax}
    \cs{Beanover}\marg{key--value list}
  \end{syntax}
\end{function}
The keys are the slide lists names, they are case sensitive and must contain no spaces nor `|/|' character.
When the same key is used multiple times, only the last one is taken into account.
Possible values are the \emph{range specifiers} \meta{first},
\meta{first}:\meta{length}, \meta{first}::\meta{last}, :\meta{length}::\meta{last}
where \meta{first}, \meta{length} and \meta{last} are algebraic expression involving any named overlay specification defined next when an integer.

A comma separated list of such specifiers between square brackets is also allowed, which results in a \emph{named slide list}.

The \cs{Beanover} command is used at the very beginning of the |frame| environment and thus only apply to this frame. It can be used there mutliple times. 
%
\subsection{Defining named slide lists}
The declaration
|\Beanover{|\meta{name}|=[|\meta{spec_1}|,|\meta{spec_2}|,...,|\meta{spec_n}|]}|
is a convenient shortcut for
|\Beanover{|\meta{name}|.1=|\meta{spec_1}|, |\meta{name}|.2=|\meta{spec_2}|,..., |\meta{name}|.n=|\meta{spec_n}|}|.
The rule of the previous section can apply.
%
\section{Named overlay specifications}
%
\subsection{Named slide ranges}
For named slide ranges, the named overlay specifications are detailled in the tables below together with
their replacement meaning value as \pkg{beamer} standard overlay specification.
%
\begin{center}
\begin{tabular}{>{ \ttfamily \bfseries }l|>{ \ttfamily }l}
\hline
\bfseries \textrm{syntax} & \bfseries \textrm{meaning} 
\\\hline
\multicolumn{2}{c}{\ttfamily \meta{name} = [\(i\), \(i+1\), \(i+2\),...]}
\\\hline
\meta{name}.1 & \(i\)\\
\meta{name}.2 & \(i+1\) \\
\meta{name}.\meta{integer} & \(i+ \meta{integer} - 1\) \\
\hline
\end{tabular}
\end{center}
In the frame example below, we use the \cs{BeanoverEval} command for the demonstration.
It is mainly used for debugging and testing purposes.

\begin{tcblisting} {
  listing only,
}
 \begin{frame} {Frame \insertframenumber} {Slide \insertslidenumber}
 \Beanover{
   A = 3,
 }
 \ttfamily
 \BeanoverEval(A.1) ==3,
 \BeanoverEval(A.2) ==4,
 \BeanoverEval(A.-1)==1,
 \end{frame}
 \end{tcblisting}
%
When the slide range has been given a length or an end,
like in the frame example below, we also have
\begin{center}
\begin{tabular}{>{ \ttfamily \bfseries }l|>{ \ttfamily }l|l|l}
\hline
\bfseries \textrm{syntax} & \bfseries \textrm{meaning} & & \bfseries \textrm{output}
\\\hline
\multicolumn{2}{c|}{\ttfamily \meta{name} = [\(i\), \(i+1\),..., \(j\)]}
\\\hline
\meta{name}.length & \(j-i+1\) & A.length & 6\\
\meta{name}.last & \(j\) & A.last & 8\\
\meta{name}.next & \(j+1\) & A.next & 9\\
\meta{name}.range & \(i\) ''-'' \(j\) & A.range & 3-8\\
\hline
\end{tabular}
\end{center}
\begin{tcblisting} {
  listing only,
}
 \begin{frame} {Frame \insertframenumber} {Slide \insertslidenumber}
 \Beanover{
   A = 3:6,
 }
 \ttfamily
 \BeanoverEval(A.length) == 6,
 \BeanoverEval(A.1)      == 3,
 \BeanoverEval(A.2)      == 4,
 \BeanoverEval(A.-1)     == 1,
 \end{frame}
 \end{tcblisting}
%
Using these specification on unfinite named slide ranges is unsupported.
%
Finally each named slide range has a dedicated counter \texttt{\meta{name}.n}
which is some kind of variable that can be used and incremented.
\begin{description}
\item[\ttfamily\meta{name}.n]: use the position of the counter 
\item[\ttfamily\meta{name}.n+=\meta{integer}]: advance the counter by \meta{integer} and use the new position
\item[\ttfamily++\meta{name}.n]: advance the counter by 1 and use the new position
\end{description}
Notice that |.n| can generally be omitted.
%
\subsection{Named slide lists}
The declaration
|\Beanover{A=[|\meta{spec_1}|,|\meta{spec_2}|,...,|\meta{spec_n}|]}|
is a convenient shortcut for
|\Beanover{A.1=|\meta{spec_1}|, A.2=|\meta{spec_2}|,..., A.n=|\meta{spec_n}|}|.
The rule of the previous section can apply.
%
\section{Named overlay specifications}
%
For named slide ranges, the named overlay specifications are detailled in the tables below together with
their replacement meaning value as \pkg{beamer} standard overlay specification.
%
\section{\texttt{?(...)} query expressions}
\pkg{beamer} defines \meta{overlay specifications} included between pointed brackets. Before they are processed by the \pkg{beamer} class,
the \pkg{beanover} package scans the \meta{overlay specifications} for any occurrence of `\texttt{?(\meta{queries})}'. Each of them is then evaluated and replaced by its static counterpart.
The overall result is finally forwarded to \pkg{beamer}.

The \meta{queries} argument is a comma separated list of individual \meta{query}'s of next table.
Sometimes, using \texttt{\meta{name}.range} is not allowed as it would lead to
an algeabraix difference instead of a range.
\begin{center}
\begin{tabular}{>{ \ttfamily \bfseries }l|>{ \ttfamily }l|>{ \ttfamily }l}
\hline
\bfseries \textrm{query} & \bfseries \textrm{static value} & \bfseries \textrm{limitation}
\\\hline
|:| & `-'\\
|::| & `-'\\
\meta{first expr} & \meta{first}\\
\meta{first expr}|:| & \meta{first} '-' & no \texttt{ \meta{name}.range }\\
\meta{first expr}|::| & \meta{first} '-' & no \texttt{ \meta{name}.range }\\
\meta{first expr}|:|\meta{length expr} & \meta{first} '-' \meta{last} & no \texttt{ \meta{name}.range }\\
\meta{first expr}|:|\meta{end expr} & \meta{first} '-' \meta{last}  & no \texttt{ \meta{name}.range }\\
\hline
\end{tabular}
\end{center}
Here \meta{first expr}, \meta{length expr} and \meta{end expr}
both denote algebraic expressions possibly involving named overlay specifications and counters.
As integers, they respectively evaluate to \meta{first}, \meta{length} and \meta{last}.

For example both \texttt{?(A.next)}, \texttt{?(A.last+1)}, \texttt{?(A.1+A.length)} give the same result as soon as the slide range named `|A|' has been defined with a length.

\end{documentation}
\DocInput{beanover.dtx}
\end{document}
%</driver>
% \fi
%
%\StopEventually{^^A
%  \PrintChanges
%  \PrintIndex
%}
%
%    \begin{macrocode}
%<*package>
%    \end{macrocode}
%
% \section{Implementation}
%
% Identify the internal prefix (\LaTeX3 \pkg{DocStrip} convention).
%    \begin{macrocode}
%<@@=beanover>
%    \end{macrocode}
%
% \subsection{Package declarations}
%
%    \begin{macrocode}
\NeedsTeXFormat{LaTeX2e}[2020/01/01]
\ProvidesExplPackage
  {beanover}
  {2022/10/28}
  {1.0}
  {Named overlay specifications for beamer}
%    \end{macrocode}
%
% \begin{implementation}
% \ExplSyntaxOn
% \NewDocumentEnvironment{test}{}
%   {\color{red}\bfseries\ExplSyntaxOn\__beanover_gclear:}
%   {\__beanover_gclear:\ExplSyntaxOff\ignorespacesafterend}
% \ExplSyntaxOff
%
% \NewDocumentEnvironment{myFailure}{}
%   {\begin{list}{\llap{\emoji{no-entry}\quad}}{%
%      \setlength{\itemindent}{0pt}%
%      \setlength{\leftmargin}{0pt}%
%      \setlength{\labelwidth}{-1\parindent}%
%      \addtolength{\topsep}{-0.5\parskip}%
%      \listparindent \parindent
%      \setlength{\parsep}{\parskip}}}
%   {\end{list}}
%
% \NewDocumentEnvironment{myList}{}
%   {\begin{list}{•}{%
%      \setlength{\itemindent}{0pt}%
%      \setlength{\leftmargin}{0pt}%
%      \setlength{\labelwidth}{-1\parindent}%
%      \addtolength{\topsep}{-0.5\parskip}%
%      \listparindent \parindent
%      \setlength{\parsep}{\parskip}}}
%   {\end{list}}
%
% \begin{test}
% \cs_gset:Npn \myDebug #1 {
%   \begin{list}{\llap{\emoji{no-entry}\quad}}{%
%      \setlength{\itemindent}{0pt}%
%      \setlength{\leftmargin}{0pt}%
%      \setlength{\labelwidth}{-1\parindent}%
%      \addtolength{\topsep}{-0.5\parskip}%
%      \listparindent \parindent
%      \setlength{\parsep}{\parskip}}
%   \clist_map_inline:nn { #1 } {
%     \item \exp_args:Nx \use:n { ##1 }
%   }
%   \end{list}
% }
% \cs_gset:Npn \__beanover_test_extract:nnnn #1 #2 #3 #4 {
%   \regex_extract_once:nnNTF { #1 } { #2 } \l_match_seq {
%     \tl_set:Nx \l_a_tl { \seq_use:Nn \l_match_seq , }
%     \seq_set_from_clist:Nn \l_split_seq { #3 }
%     \tl_set:Nx \l_b_tl { \seq_use:Nn \l_split_seq , }
%     \exp_args:Nxx \tl_if_eq:nnF { \l_a_tl } { \l_b_tl } {
%       \begin{myFailure}
%       \item FAILURE~`\l_a_tl'!=`\l_b_tl'
%       \item #4~(#2)
%       \end{myFailure}
%     }
%   } {
%     \begin{myFailure}{}
%     \item FAILURE~NO~MATCH
%     \item #4~(#2)
%     \end{myFailure}
%   }
% }
% \cs_gset:Npn \__beanover_test_extract:Nnnn #1 #2 #3 #4 {
%   \regex_extract_once:NnNTF #1 { #2 } \l_match_seq {
%     \tl_set:Nx \l_a_tl { \seq_use:Nn \l_match_seq , }
%     \seq_set_from_clist:Nn \l_split_seq { #3 }
%     \tl_set:Nx \l_b_tl { \seq_use:Nn \l_split_seq , }
%     \exp_args:Nxx \tl_if_eq:nnF { \l_a_tl } { \l_b_tl } {
%       \begin{myFailure}
%       \item FAILURE~`\l_a_tl'!=`\l_b_tl'
%       \item #4~(#2)
%       \end{myFailure}
%     }
%   } {
%     \begin{myFailure}
%     \item FAILURE~NO~MATCH
%     \item #4~(#2)
%     \end{myFailure}
%   }
% }
% \cs_gset:Npn \__beanover_test_split:Nnnn #1 #2 #3 #4 {
%   \regex_split:NnN #1 { #2 } \l_match_seq
%   \tl_set:Nx \l_a_tl { \seq_use:Nn \l_match_seq , }
%   \seq_set_from_clist:Nn \l_split_seq { #3 }
%   \tl_set:Nx \l_b_tl { \seq_use:Nn \l_split_seq , }
%   \exp_args:Nxx \tl_if_eq:nnF { \l_a_tl } { \l_b_tl } {
%     \begin{myFailure}
%     \item FAILURE~`\l_a_tl'!=`\l_b_tl'
%     \item Test~split~#4~(#2)
%     \end{myFailure}
%   }
% }
% \cs_gset:Npn \__beanover_assert_equal:nnn #1 #2 #3 {
%   \tl_if_eq:nnF { #1 } { #2 } {
%     \begin{myFailure}
%     \item FAILURE~`#1'!=`#2'
%     \item #3
%     \end{myFailure}
%   }
% }
% \cs_generate_variant:Nn \__beanover_assert_equal:nnn { xxn }
% \end{test}
%
% \subsection{Local variables}
% We make heavy use of local variables and function scopes.
% Many functions are executed within a \TeX\ group,
% which ensures no name collision with the caller stack.
% In that case, variables need not follow exactly the \LaTeX3\ naming convention:
% we do not specialize with the module name.
% On execution, next group initialization instructions declare the variables as side effect.
%    \begin{macrocode}
\group_begin:
\tl_clear_new:N  \l_a_tl
\tl_clear_new:N  \l_b_tl
\tl_clear_new:N  \l_c_tl
\tl_clear_new:N  \l_ans_tl
\seq_clear_new:N \l_ans_seq
\seq_clear_new:N \l_match_seq
\seq_clear_new:N \l_token_seq
\int_zero_new:N  \l_split_int
\seq_clear_new:N \l_split_seq
\int_zero_new:N  \l_depth_int
\tl_clear_new:N  \l_name_tl
\tl_clear_new:N  \l_path_tl
\tl_clear_new:N  \l_group_tl
\tl_clear_new:N  \l_query_tl
\seq_clear_new:N \l_query_seq
\bool_set_false:N \l_no_counter_bool
\bool_set_false:N \l_no_range_bool
\group_end:
%    \end{macrocode}
%
% \subsection{Overlay specification}
% \subsubsection{In slide range definitions}
% \begin{variable} {\g_@@_prop}
% \meta{key}--\meta{value} property list to store the named slide lists.
% The basic keys are, assuming \meta{name} is a slide list identifier,
% \begin{description}
% \item [\texttt{\meta{name}/A}] for the first index
% \item [\texttt{\meta{name}/L}] for the length when provided
% \item [\texttt{\meta{name}/Z}] for the last index when provided
% \item [\texttt{\meta{name}/C}] for the counter value, when used
% \item [\texttt{\meta{name}/C0}] for initial value of the counter (when reset)
% \end{description}
% Other keys are eventually used to cache results when some attributes
% are defined from other slide ranges. They are characterized by a `|//|'.
% \begin{description}
% \item [\texttt{\meta{name}//A}] for the cached static value of the first index
% \item [\texttt{\meta{name}//Z}] for the cached static value of the last index
% \item [\texttt{\meta{name}//L}] for the cached static value of the length
% \item [\texttt{\meta{name}//N}] for the cached static value of the next index
% \end{description}
% The implementation is private, in particular, keys may change in future versions.
%    \begin{macrocode}
\prop_new:N \g_@@_prop
%    \end{macrocode}
% \end{variable}
% \begin{function}{\@@_gput:nn,\@@_gput:nV,\@@_item:n,\@@_get:nN,\@@_gremove:n,\@@_gclear:n,\@@_gclear:}
% \begin{syntax}
% \cs{@@_gput:nn} \marg{key} \marg{value}
% \cs{@@_item:n} \marg{key}
% \cs{@@_get:n} \marg{key} \meta{tl variable}
% \cs{@@_gremove:n} \marg{key}
% \cs{@@_gclear:n} \marg{key}
% \cs{@@_gclear:}
% \end{syntax}
% Convenient shortcuts to manage the storage, it makes the code more concise and readable.
%    \begin{macrocode}
\cs_new:Npn \@@_gput:nn {
  \prop_gput:Nnn \g_@@_prop
}
\cs_new:Npn \@@_item:n {
  \prop_item:Nn \g_@@_prop
}
\cs_new:Npn \@@_get:nN {
  \prop_get:NnN \g_@@_prop
}
\cs_new:Npn \@@_gremove:n {
  \prop_gremove:Nn \g_@@_prop
}
\cs_new:Npn \@@_gclear:n #1 {
  \clist_map_inline:nn { A, L, Z, C, C0, /A, /L, /Z, /N } {
    \@@_gremove:n { #1 / ##1 }
  }
}
\cs_new:Npn \@@_gclear: {
  \prop_gclear:N \g_@@_prop
}
\cs_generate_variant:Nn \@@_gput:nn { nV }
%    \end{macrocode}
% \end{function}
% \begin{function}[pTF]{\@@_if_in:n,\@@_if_in:V}
% \begin{syntax}
% \cs{@@_if_in_p:n} \marg{key}
% \cs{@@_if_in:nTF} \marg{key} \Arg{true code} \Arg{false code}
% \end{syntax}
% Convenient shortcuts to test for the existence of some key, it makes the code more concise and readable.
%    \begin{macrocode}
\prg_new_conditional:Npnn \@@_if_in:n #1 { p, T, F, TF } {
  \prop_if_in:NnTF \g_@@_prop { #1 } {
    \prg_return_true:
  } {
    \prg_return_false:
  }
}
\prg_generate_conditional_variant:Nnn \@@_if_in:n {V} { p, T, F, TF }
%    \end{macrocode}
% \end{function}
% \begin{function}[TF]{\@@_get:nN}
% \begin{syntax}
% \cs{@@_get:nNTF} \marg{key} \meta{tl variable} \Arg{true code} \Arg{false code}
% \end{syntax}
% Convenient shortcuts to retrieve the value with branching, it makes the code more concise and readable.
% Execute \meta{true code} when the item is found, \meta{false code} otherwise.
% In the latter case, the content of the \meta{tl variable} is undefined.
%    \begin{macrocode}
\prg_new_conditional:Npnn \@@_get:nN #1 #2 { T, F, TF } {
  \prop_get:NnNTF \g_@@_prop { #1 } #2 {
    \prg_return_true:
  } {
    \prg_return_false:
  }
}
%    \end{macrocode}
% \end{function}
% \begin{test}
% \cs_set:Npn \Test:xxn #1 #2 #3 {
%   \__beanover_assert_equal:xxn { #1 } { #2 } {Test~\cs{__beanover_g(put|remove|clear|...):nn}~#3}
% }
% \__beanover_gput:nn { A } { B }
% \Test:xxn { \__beanover_item:n {A} } { B } { 1 }
% \Test:xxn { \__beanover_if_in:nTF {A} TF } { T } { 3 }
% \Test:xxn { \__beanover_if_in:nTF {B} TF } { F } { 4 }
% \Test:xxn { \__beanover_if_in:nT {A} T } { T } { 5 }
% \Test:xxn { \__beanover_if_in:nT {B} T } { } { 6 }
% \Test:xxn { \__beanover_if_in:nF {A} F } { } { 7 }
% \Test:xxn { \__beanover_if_in:nF {B} F } { F } { 8 }
% \Test:xxn { \bool_if:nTF { \__beanover_if_in_p:n {A} } TF } { T } { 9 }
% \Test:xxn { \bool_if:nTF { \__beanover_if_in_p:n {B} } TF } { F } { 10 }
% \Test:xxn { \bool_if:nTF { \__beanover_if_in_p:n {B} || \__beanover_if_in_p:n {A} } TF } { T } { 11 }
% \tl_clear:N \l_b_tl
% \__beanover_get:nNTF {A} \l_b_tl {
%   \tl_set:Nn \l_b_tl { SUCCESS }
% } {
%   \tl_set:Nn \l_b_tl { FAILURE }
% }
% \Test:xxn { \l_b_tl } { SUCCESS } { 13 }
% \__beanover_get:nNTF {B} \l_b_tl {
%   \tl_set:Nn \l_b_tl { FAILURE }
% } {
%   \tl_set:Nn \l_b_tl { SUCCESS }
% }
% \Test:xxn { \l_b_tl } { SUCCESS } { 13 }
% \end{test}
% Utility message.
%    \begin{macrocode}
\msg_new:nnn { beanover } { :n } { #1 }
%    \end{macrocode}
%
% \subsubsection{Regular expressions}
%
% \begin{variable}{\c@@_name_regex}
% The name of a slide range consists of a non void list of alphanumerical characters and
% underscore, but with no leading digit.
%    \begin{macrocode}
\regex_const:Nn \c@@_name_regex {
  [[:alpha:]_][[:alnum:]_]*
}
%    \end{macrocode}
% \begin{test}
% \__beanover_test_extract:Nnnn \c__beanover_name_regex {A} {A} {Test~1}
% \__beanover_test_extract:Nnnn \c__beanover_name_regex {_A1} {_A1} {Test~2}
% \end{test}
% \end{variable}
% \begin{variable}{\c@@_path_regex}
% A sequence of \texttt{.\meta{positive integer}} items representing a path.
%    \begin{macrocode}
\regex_const:Nn \c@@_path_regex {
  (?: \. \d+ )*
}
%    \end{macrocode}
% \begin{test}
% \__beanover_test_extract:Nnnn \c__beanover_path_regex {} {} {Test~1}
% \__beanover_test_extract:Nnnn \c__beanover_path_regex {.1} {.1} {Test~2}
% \__beanover_test_extract:Nnnn \c__beanover_path_regex {.1.2} {.1.2} {Test~3}
% \end{test}
% \end{variable}
%
% \begin{variable}{\c@@_key_regex, \c@@_A_key_Z_regex}
% A key is the name of a slide range possibly followed by positive integer attributes
% using a dot syntax.
% The `|A_key_Z|' variant matches the whole string.
%    \begin{macrocode}
\regex_const:Nn \c@@_key_regex {
  \ur{c@@_name_regex} \ur{c@@_path_regex}
}
\regex_const:Nn \c@@_A_key_Z_regex { 
  \A \ur{c@@_key_regex} \Z
}
%    \end{macrocode}
% \begin{test}
% \__beanover_test_extract:Nnnn \c__beanover_key_regex {A} {A} {Test~1}
% \__beanover_test_extract:Nnnn \c__beanover_key_regex {_A1} {_A1} {Test~2}
% \__beanover_test_extract:Nnnn \c__beanover_key_regex {A.1} {A.1} {Test~3}
% \__beanover_test_extract:Nnnn \c__beanover_key_regex {_A1.1} {_A1.1} {Test~4}
% \__beanover_test_extract:Nnnn \c__beanover_key_regex {_A1.1.2} {_A1.1.2} {Test~5}
% \__beanover_test_extract:Nnnn \c__beanover_key_regex {(_A1.1.2)} {_A1.1.2} {Test~6}
% \__beanover_test_extract:Nnnn \c__beanover_A_key_Z_regex {A} {A} {Test~1}
% \__beanover_test_extract:Nnnn \c__beanover_A_key_Z_regex {_A1} {_A1} {Test~2}
% \__beanover_test_extract:Nnnn \c__beanover_A_key_Z_regex {A.1} {A.1} {Test~3}
% \__beanover_test_extract:Nnnn \c__beanover_A_key_Z_regex {_A1.1} {_A1.1} {Test~4}
% \__beanover_test_extract:Nnnn \c__beanover_A_key_Z_regex {_A1.1.2} {_A1.1.2} {Test~5}
% \end{test}
% \end{variable}
%
% \begin{variable}{\c@@_dotted_regex}
% A specifier is the name of a slide range possibly followed by attributes
% using a dot syntax.
% This is a poor man version to save computations,
% a dedicated parser would help in error management.
%    \begin{macrocode}
\regex_const:Nn \c@@_dotted_regex {
  \A \ur{c@@_name_regex} (?: \. [^.]+ )* \Z
}
%    \end{macrocode}
% \begin{test}
% \__beanover_test_extract:Nnnn \c__beanover_dotted_regex {A} {A} {Test~1}
% \__beanover_test_extract:Nnnn \c__beanover_dotted_regex {_A1} {_A1} {Test~2}
% \__beanover_test_extract:Nnnn \c__beanover_dotted_regex {A.1} {A.1} {Test~3}
% \__beanover_test_extract:Nnnn \c__beanover_dotted_regex {_A1.-1} {_A1.-1} {Test~4}
% \__beanover_test_extract:Nnnn \c__beanover_dotted_regex {_A1.-1.2} {_A1.-1.2} {Test~5}
% \__beanover_test_extract:Nnnn \c__beanover_dotted_regex {_A1.range} {_A1.range} {Test~6}
% \end{test}
% \end{variable}
%
% \begin{variable}{\c@@_colons_regex}
% For ranges defined by a colon syntax.
%    \begin{macrocode}
\regex_const:Nn \c@@_colons_regex { :(:+)? }
%    \end{macrocode}
% \begin{test}
% \__beanover_test_split:Nnnn \c__beanover_colons_regex
%   { A:C } {{A},{},{C}} {Test~1}
% \__beanover_test_split:Nnnn \c__beanover_colons_regex
%   { A::C } {{A},{:},{C}} {Test~2}
% \__beanover_test_split:Nnnn \c__beanover_colons_regex
%   { A:::C } {{A},{::},{C}} {Test~3}
% \__beanover_test_split:Nnnn \c__beanover_colons_regex
%   { :B::C } {{},{},{B},{:},{C}} {Test~4}
% \end{test}
% \end{variable}
%
% \begin{variable}{\c@@_int_regex}
% A decimal integer with an eventual leading sign next to the first digit.
%    \begin{macrocode}
\regex_const:Nn \c@@_int_regex {
  (?:[-+])? \d+
}
%    \end{macrocode}
% \begin{test}
% \__beanover_test_extract:Nnnn \c__beanover_int_regex
%   {421} {421} {Test~1}
% \__beanover_test_extract:Nnnn \c__beanover_int_regex
%   {+421} {+421} {Test~2}
% \__beanover_test_extract:Nnnn \c__beanover_int_regex
%   {-421} {-421} {Test~3}
% \end{test}
% \end{variable}
%
% \begin{variable}{\c@@_list_regex}
% A comma separated list between square brackets.
%    \begin{macrocode}
\regex_const:Nn \c@@_list_regex {
  \A \[ \s*
%    \end{macrocode}
% Capture groups:
% \begin{myList}
% \item 2: the content between the brackets, outer spaces trimmed out
%    \begin{macrocode}
    ( [^\]]*? )
  \s* \] \Z
}
%    \end{macrocode}
% \end{myList}
% \begin{test}
% \__beanover_test_extract:Nnnn \c__beanover_list_regex
%   {[]} {[],{}} {Test~1}
% \__beanover_test_extract:Nnnn \c__beanover_list_regex
%   {[x]} {[x], x} {Test~2}
% \__beanover_test_extract:Nnnn \c__beanover_list_regex
%   {[~x~]} {[~x~],{x}} {Test~3}
% \__beanover_test_extract:Nnnn \c__beanover_list_regex
%   {[~x[~]} {[~x[~],{x[}} {Test~4}
% 
% \end{test}
% \end{variable}
%
% \begin{variable}{\c@@_split_regex}
% Used to parse slide list overlay specifications in queries.
% Next are the 10 capture groups.
% Group numbers are 1 based because the regex is used in splitting contexts
% where only capture groups are considered and not the whole match.
%    \begin{macrocode}
\regex_const:Nn \c@@_split_regex {
  \s* ( ? :
%    \end{macrocode}
% We start with `|++|' instrussions
% \footnote{At the same time an instruction and an expression... this is a synonym of exprection}.
%   \begin{myList}
%   \item 1: \meta{name} of a slide range
%    \begin{macrocode}
    \+\+ ( \ur{c@@_name_regex} )
%    \end{macrocode}
%   \item 2: optionally followed by an integer path
%    \begin{macrocode}
    ( \ur{c@@_path_regex} ) (?: \. n )?
%    \end{macrocode}
% \begin{test}
% \__beanover_test_extract:Nnnn \c__beanover_split_regex
%   { ++ABC.n } {{++ABC.n},{ABC},{},{},{},{},{},{},{},{},{}} {Test~2a}
% \__beanover_test_extract:Nnnn \c__beanover_split_regex
%   { ++ABC.1.2.n~~ } {{++ABC.1.2.n~~},{ABC},{.1.2},{},{},{},{},{},{},{},{}} {Test~2b}
% \__beanover_test_extract:Nnnn \c__beanover_split_regex
%   { ++ABC } {{++ABC},{ABC},{},{},{},{},{},{},{},{},{}} {Test~2c}
% \__beanover_test_extract:Nnnn \c__beanover_split_regex
%   { ++ABC.1.2~~ } {{++ABC.1.2~~},{ABC},{.1.2},{},{},{},{},{},{},{},{}} {Test~2d}
% \end{test}
% We continue with other expressions
%   \item 3: \meta{name} of a slide range
%    \begin{macrocode}
  | ( \ur{c@@_name_regex} )
%    \end{macrocode}
%   \item 4: optionally followed by an integer path
%    \begin{macrocode}
    ( \ur{c@@_path_regex} )
%    \end{macrocode}
% Next comes another branching
%    \begin{macrocode}
    (?:
%    \end{macrocode}
%   \item 5: the \meta{length} attribute
%    \begin{macrocode}
       \. l(e)ngth
%    \end{macrocode}
% \begin{test}
% \__beanover_test_extract:Nnnn \c__beanover_split_regex
%   { A.length } {{A.length},{},{},{A},{},{e},{},{},{},{},{}} {Test~5a}
% \__beanover_test_extract:Nnnn \c__beanover_split_regex
%   { A.1.2.length } {{A.1.2.length},{},{},{A},{.1.2},{e},{},{},{},{},{}} {Test~5b}
% \end{test}
%   \item 6: the \meta{last} attribute
%    \begin{macrocode}
    |  \. l(a)st
%    \end{macrocode}
% \begin{test}
% \__beanover_test_extract:Nnnn \c__beanover_split_regex
%   { A.last } {{A.last},{},{},{A},{},{},{a},{},{},{},{}} {Test~6a}
% \__beanover_test_extract:Nnnn \c__beanover_split_regex
%   { A.1.2.last } {{A.1.2.last},{},{},{A},{.1.2},{},{a},{},{},{},{}} {Test~6b}
% \end{test}
%   \item 7: the \meta{next} attribute
%    \begin{macrocode}
    |  \. ne(x)t
%    \end{macrocode}
% \begin{test}
% \__beanover_test_extract:Nnnn \c__beanover_split_regex
%   { A.next } {{A.next},{},{},{A},{},{},{},{x},{},{},{}} {Test~7a}
% \__beanover_test_extract:Nnnn \c__beanover_split_regex
%   { A.1.2.next } {{A.1.2.next},{},{},{A},{.1.2},{},{},{x},{},{},{}} {Test~7b}
% \end{test}
%   \item 8: the \meta{range} attribute
%    \begin{macrocode}
    |  \. (r)ange
%    \end{macrocode}
% \begin{test}
% \__beanover_test_extract:Nnnn \c__beanover_split_regex
%   { A.range } {{A.range},{},{},{A},{},{},{},{},{r},{},{}} {Test~8a}
% \__beanover_test_extract:Nnnn \c__beanover_split_regex
%   { A.1.2.range } {{A.1.2.range},{},{},{A},{.1.2},{},{},{},{r},{},{}} {Test~8b}
% \end{test}
%   \item 9: the \meta{n} attribute
%    \begin{macrocode}
    |  \. (n)
%    \end{macrocode}
% \begin{test}
% \__beanover_test_extract:Nnnn \c__beanover_split_regex
%   { A.n } {{A.n},{},{},{A},{},{},{},{},{},{n},{}} {Test~9a}
% \__beanover_test_extract:Nnnn \c__beanover_split_regex
%   { A.1.2.n } {{A.1.2.n},{},{},{A},{.1.2},{},{},{},{},{n},{}} {Test~9b}
% \end{test}
%   \item 10: the poor man integer expression after `|+=|'.
%   When it contains no parenthesis, it is an algebraic expression involving
%   integers and \meta{key}'s.
%   Otherwise it starts with a parenthesis and ends with the first parenthesis
%   followed by a white space or the end of the text.
%   This tricky definition allows quite any algebraic expression involving parenthesis.
%   The problems may arise when dealing with nested expressions.
%    \begin{macrocode}
        (?: \s* \+= \s*
          ( (?: \ur{c@@_int_regex} | \ur{c@@_key_regex} )
            (?: [+\-*/] (?: \d+ | \ur{c@@_key_regex}) )*
          | \( .*? \) (?: \Z | \s+ )
          )
        )?
%    \end{macrocode}
% \begin{test}
% \__beanover_test_extract:Nnnn \c__beanover_split_regex
%   { ABC.n += 421 } {{ABC.n += 421},{},{},{ABC},{},{},{},{},{},{n},{421}} {Test~`+='~1}
% \__beanover_test_extract:Nnnn \c__beanover_key_regex
%   { P.1 } {P.1} {Test~`+='~2}
% \__beanover_test_extract:Nnnn \c__beanover_split_regex
%   { ABC.n += 1+P.1 } {{ABC.n += 1+P.1},{},{},{ABC},{},{},{},{},{},{n},{1+P.1}} {Test~`+='~2}
% \__beanover_test_extract:Nnnn \c__beanover_split_regex
%   { ABC.n += 1+P.1X } {{ABC.n += 1+P.1},{},{},ABC,{},{},{},{},{}, {n},{1+P.1}} {Test~`+='~3}
% \__beanover_test_extract:Nnnn \c__beanover_split_regex
%   { ABC.n += (P.1+1)~~^^A(
% ) } {{ABC.n += (P.1+1)~},{},{},ABC,{},{},{},{},{}, {n},{(P.1+1)~}} {Test~`+='~4}
% \end{test}
%    \begin{macrocode}
    )?
%    \end{macrocode}
% \end{myList}
%    \begin{macrocode}
  ) \s*
}
%    \end{macrocode}
% \end{variable}
%
% \subsubsection{Defining named slide ranges}
%
% \begin{function}{\@@_error:n}
% Prints an error message when a key only item is used.
%    \begin{macrocode}
\cs_new:Npn \@@_error:n #1 {
  \msg_fatal:nnn { beanover } { :n } { Missing~value~for~#1 }
}
%    \end{macrocode}
% \end{function}
% \begin{function}{\@@_parse:nn}
%   \begin{syntax}
%     \cs{@@_parse:nn} \marg{key} \marg{definition}
%   \end{syntax}
% Auxiliary function called within a group.
% \meta{name} is the slide key, 
% including eventually a dotted integer path,
% \meta{definition} is the corresponding definition.
% \begin{variable}{\l_match_seq}
% Local storage for the match result.
% \end{variable}
% \end{function}
%
% \begin{function}{\@@_range:nnnn, \@@_range:nVVV,\@@_range:nnnnn, \@@_range:nVVVV}
%   \begin{syntax}
%     \cs{@@_range:nnnn}  \marg{key} \marg{first} \marg{length} \marg{last}
%     \cs{@@_range:nnnnn} \marg{name} \marg{path} \marg{first} \marg{length} \marg{last}
%   \end{syntax}
% Auxiliary function called within a group.
% Setup the model to define a range.
%    \begin{macrocode}
\cs_new:Npn \@@_range:nnnn #1 #2 #3 #4 {
  \@@_gclear:n { #1 }
  \tl_if_empty:nTF { #2 } {
    \tl_if_empty:nTF { #3 } {
      \tl_if_empty:nTF { #4 } {
        \msg_error:nnn { beanover } { :n } { Not~a~range:~:~#1 }
      } {
        \@@_gput:nn { #1/Z } { #4 }
      }
    } {
      \@@_gput:nn { #1/L } { #3 }
      \tl_if_empty:nF { #4 } {
        \@@_gput:nn { #1/Z } { #4 }
        \@@_gput:nn { #1/A } { #1.last - (#1.length) + 1 }
      }
    }
  } {
    \@@_gput:nn { #1/A } { #2 }
    \tl_if_empty:nTF { #3 } {
      \tl_if_empty:nF { #4 } {
        \@@_gput:nn { #1/Z } { #4 }
        \@@_gput:nn { #1/L } { #1.last - (#1.first) + 1 }
      }
    } {
      \@@_gput:nn { #1/L } { #3 }
      \@@_gput:nn { #1/Z } { #1.first + #1.length - 1 }
    }
  }
}
\cs_generate_variant:Nn \@@_range:nnnn { nVVV }
%    \end{macrocode}%
% \begin{test}
% \cs_set:Npn \Test #1 #2 #3 #4 #5 {
%   \__beanover_assert_equal:xxn {\__beanover_item:n {#1/A} } { #2 } {Test~\cs{__beanover_range:nnnn}~#5~A }
%   \__beanover_assert_equal:xxn {\__beanover_item:n {#1/L} } { #3 } {Test~\cs{__beanover_range:nnnn}~#5~L }
%   \__beanover_assert_equal:xxn {\__beanover_item:n {#1/Z} } { #4 } {Test~\cs{__beanover_range:nnnn}~#5~Z }
% }
% \__beanover_range:nnnn {X} {FIRST} {} {}
% \Test { X } {FIRST} {} {} {TestB~\cs{__beanover_range:nnnn}~X}
% \__beanover_range:nnnn {X} {} {LENGTH} {}
% \Test { X } {} {LENGTH} {} {TestC~\cs{__beanover_range:nnnn}~X.1}
% \__beanover_range:nnnn {X} {} {} {LAST}
% \Test { X } {} {} {LAST} {TestC~\cs{__beanover_range:nnnn}~X.1}
% \__beanover_range:nnnn {X.1} {FIRST} {} {}
% \Test { X.1 } {FIRST} {} {} {TestB~\cs{__beanover_range:nnnn}~X.1}
% \__beanover_range:nnnn {X.1} {} {LENGTH} {}
% \Test { X.1 } {} {LENGTH} {} {TestC~\cs{__beanover_range:nnnn}~X.1}
% \__beanover_range:nnnn {X.1} {} {} {LAST}
% \Test { X.1 } {} {} {LAST} {TestC~\cs{__beanover_range:nnnn}~X.1}
% \__beanover_range:nnnn {X} {FIRST} {LENGTH} {}
% \Test { X } {FIRST} {LENGTH} {X.first+X.length-1} {Test~\cs{__beanover_range:nnnn}~a}
% \__beanover_range:nnnn {X} {FIRST} {} {LAST}
% \Test { X } {FIRST} {X.last-(X.first)+1} {LAST} {Test~\cs{__beanover_range:nnnn}~b}
% \__beanover_range:nnnn {X} {} {LENGTH} {LAST}
% \Test { X } {X.last-(X.length)+1} {LENGTH} {LAST} {Test~\cs{__beanover_range:nnnn}~c}
% \cs_set:Npn \msg_error:nnn #1 #2 #3 {
%   \cs_set:Npn \Test ##1 ##2 ##3 ##4 ##5 { }
% }
% \cs_set:Npn \Test #1 #2 #3 #4 #5 {
%   \__beanover_assert_equal:xxn { MISSING~ERROR } {  } {Test~\cs{__beanover_range:nnnn}~#5~Z }
% }
% \__beanover_range:nnnn {X} {} {} {}
% \Test { X } {} {} {} {TestA~\cs{__beanover_range:nnnn}~X}
% \cs_set:Npn \Test #1 #2 #3 #4 #5 {
%   \__beanover_assert_equal:xxn { MISSING~ERROR } {  } {Test~\cs{__beanover_range:nnnn}~#5~Z }
% }
% \__beanover_range:nnnn {X.1} {} {} {}
% \Test { X.1 } {} {} {} {TestA~\cs{__beanover_range:nnnn}~X.1}
% \end{test}
% \end{function}
%
%    \begin{macrocode}
\cs_generate_variant:Nn \tl_if_empty:nTF { xTF }
\cs_new:Npn \@@_do_parse:nn #1 #2 {
%    \end{macrocode}
% This is not a list.
%    \begin{macrocode}
  \tl_clear:N \l_a_tl
  \tl_clear:N \l_b_tl
  \tl_clear:N \l_c_tl
  \regex_split:NnN \c@@_colons_regex { #2 } \l_split_seq
  \seq_pop_left:NNT \l_split_seq \l_a_tl {
%    \end{macrocode}
% \cs{l_a_tl} may contain the \meta{start}.
%    \begin{macrocode}
    \seq_pop_left:NNT \l_split_seq \l_b_tl {
      \tl_if_empty:NTF \l_b_tl {
%    \end{macrocode}
% This is a one colon range.
%    \begin{macrocode}
        \seq_pop_left:NN \l_split_seq \l_b_tl
%    \end{macrocode}
% \cs{l_b_tl} may contain the \meta{length}.
%    \begin{macrocode}
        \seq_pop_left:NNT \l_split_seq \l_c_tl {
          \tl_if_empty:NTF \l_c_tl {
%    \end{macrocode}
% A |::| was expected:
%    \begin{macrocode}
\msg_error:nnn { beanover } { :n } { Invalid~range~expression(1):~#2 }
          } {
            \int_compare:nNnT { \tl_count:N \l_c_tl } > { 1 } {
\msg_error:nnn { beanover } { :n } { Invalid~range~expression(2):~#2 }
            }
            \seq_pop_left:NN \l_split_seq \l_c_tl
%    \end{macrocode}
% \cs{l_c_tl} may contain the \meta{end}.
%    \begin{macrocode}
            \seq_if_empty:NF \l_split_seq {
\msg_error:nnn { beanover } { :n } { Invalid~range~expression(3):~#2 }
            }
          }
        }
      } {
%    \end{macrocode}
% This is a two colon range.
%    \begin{macrocode}
        \int_compare:nNnT { \tl_count:N \l_b_tl } > { 1 } {
\msg_error:nnn { beanover } { :n } { Invalid~range~expression(4):~#2 }
        }
        \seq_pop_left:NN \l_split_seq \l_c_tl
%    \end{macrocode}
% \cs{l_c_tl} contains the \meta{end}.
%    \begin{macrocode}
        \seq_pop_left:NNTF \l_split_seq \l_b_tl {
          \tl_if_empty:NTF \l_b_tl {
            \seq_pop_left:NN \l_split_seq \l_b_tl
%    \end{macrocode}
% \cs{l_b_tl} may contain the \meta{length}.
%    \begin{macrocode}
            \seq_if_empty:NF \l_split_seq {
\msg_error:nnn { beanover } { :n } { Invalid~range~expression(5):~#2 }
            }
          } {
\msg_error:nnn { beanover } { :n } { Invalid~range~expression(6):~#2 }
          }
        } {
          \tl_clear:N \l_b_tl
        }
      }
    }
  }
%    \end{macrocode}
% Prividing both the \meta{start}, \meta{length} and \meta{end} of a range
% is not allowed, even if they happen to be consistent.
%    \begin{macrocode}
  \bool_if:nF {
    \tl_if_empty_p:N \l_a_tl
    || \tl_if_empty_p:N \l_b_tl
    || \tl_if_empty_p:N \l_c_tl
  } {
\msg_error:nnn { beanover } { :n } { Invalid~range~expression(7):~#2 }
  }
  \@@_range:nVVV { #1 } \l_a_tl \l_b_tl \l_c_tl
}
%    \end{macrocode}
% \begin{test}
% \cs_set:Npn \Test:nnnn #1 #2 #3 #4 {
%   \__beanover_assert_equal:xxn {\__beanover_item:n {X/A} } {#1} {Test~\cs{__beanover_do_parse:nn}~#4-a}
%   \__beanover_assert_equal:xxn {\__beanover_item:n {X/L} } {#2} {Test~\cs{__beanover_do_parse:nn}~#4-b}
%   \__beanover_assert_equal:xxn {\__beanover_item:n {X/Z} } {#3} {Test~\cs{__beanover_do_parse:nn}~#4-c}
% }
% \__beanover_do_parse:nn {X} {A}
% \Test:nnnn {A} {} {} {1}
% \__beanover_do_parse:nn {X} {A:B}
% \Test:nnnn {A} {B} {X.first+X.length-1} {2}
% \__beanover_do_parse:nn {X} {A::C}
% \Test:nnnn {A} {X.last-(X.first)+1} {C} {3}
% \__beanover_do_parse:nn {X} {:B::C}
% \Test:nnnn {X.last-(X.length)+1} {B} {C} {4}
% \__beanover_do_parse:nn {X} {::C:B}
% \Test:nnnn {X.last-(X.length)+1} {B} {C} {5}
% {
%   \cs_set:Npn \msg_error:nnn #1 #2 #3 {
%     \cs_set:Npn \Test:nnnn ##1 ##2 ##3 ##4 { }
%   }
%   \__beanover_do_parse:nn {X} {A::C:B}
% }
% {
%   \cs_set:Npn \msg_error:nnn #1 #2 #3 {
%     \cs_set:Npn \Test:nnnn ##1 ##2 ##3 ##4 { }
%   }
%   \__beanover_do_parse:nn {X} {A:B::C}
% }
% \end{test}
%    \begin{macrocode}
\cs_new:Npn \@@_parse:nn #1 #2 {
  \group_begin:
  \regex_match:NnTF \c@@_A_key_Z_regex { #1 } {
%    \end{macrocode}
% We got a valid key.
%    \begin{macrocode}
    \regex_extract_once:NnNTF \c@@_list_regex { #2 } \l_match_seq {
%    \end{macrocode}
% This is a comma separated list, extract each item and go recursive.
%    \begin{macrocode}
      \exp_args:NNx
      \seq_set_from_clist:Nn \l_match_seq {
        \seq_item:Nn \l_match_seq { 2 }
      }
      \seq_map_indexed_inline:Nn \l_match_seq {
        \@@_do_parse:nn { #1.##1 } { ##2 }
      }
    } {
      \@@_do_parse:nn { #1 } { #2 }
    }
  } {
    \msg_error:nnn { beanover } { :n } { Invalid~key:~#1 }
  }
  \group_end:
}
%    \end{macrocode}
% \begin{test}
% \cs_set:Npn \Test:nnnn #1 #2 #3 #4 {
%   \__beanover_assert_equal:xxn {\__beanover_item:n {X/A} } {#1} {Test~\cs{__beanover_parse:nn}~#4-a}
%   \__beanover_assert_equal:xxn {\__beanover_item:n {X/L} } {#2} {Test~\cs{__beanover_parse:nn}~#4-b}
%   \__beanover_assert_equal:xxn {\__beanover_item:n {X/Z} } {#3} {Test~\cs{__beanover_parse:nn}~#4-c}
% }
% \__beanover_parse:nn {X} {A}
% \Test:nnnn {A} {} {} {1}
% \__beanover_parse:nn {X} {A:B}
% \Test:nnnn {A} {B} {X.first+X.length-1} {2}
% \__beanover_parse:nn {X} {A::C}
% \Test:nnnn {A} {X.last-(X.first)+1} {C} {3}
% \end{test}
% \begin{test}
% \cs_set:Npn \Test:nnnn #1 #2 #3 #4 {
%   \__beanover_assert_equal:xxn {\__beanover_item:n {X.1.2/A} } {#1} {Test~\cs{__beanover_parse:nn}~#4-a}
%   \__beanover_assert_equal:xxn {\__beanover_item:n {X.1.2/L} } {#2} {Test~\cs{__beanover_parse:nn}~#4-b}
%   \__beanover_assert_equal:xxn {\__beanover_item:n {X.1.2/Z} } {#3} {Test~\cs{__beanover_parse:nn}~#4-c}
% }
% \__beanover_parse:nn {X.1.2} {A}
% \Test:nnnn {A} {} {} {1}
% \__beanover_parse:nn {X.1.2} {A:B}
% \Test:nnnn {A} {B} {X.1.2.first+X.1.2.length-1} {2}
% \__beanover_parse:nn {X.1.2} {A::C}
% \Test:nnnn {A} {X.1.2.last-(X.1.2.first)+1} {C} {3}
% \cs_set:Npn \Test:nnnnn #1 #2 #3 #4 #5 {
%   \__beanover_assert_equal:xxn {\__beanover_item:n {X.#5/A} } {#1} {Test~\cs{__beanover_parse:nn}~#4-a}
%   \__beanover_assert_equal:xxn {\__beanover_item:n {X.#5/L} } {#2} {Test~\cs{__beanover_parse:nn}~#4-b}
%   \__beanover_assert_equal:xxn {\__beanover_item:n {X.#5/Z} } {#3} {Test~\cs{__beanover_parse:nn}~#4-c}
% }
% \__beanover_parse:nn {X.3} {[A,A:B,A::C,:B::C,::C:B]}
% \Test:nnnnn {A} {} {} {4-1} {3.1}
% \Test:nnnnn {A} {B} {X.3.2.first+X.3.2.length-1} {4-2} {3.2}
% \Test:nnnnn {A} {X.3.3.last-(X.3.3.first)+1} {C} {4-3} {3.3}
% \Test:nnnnn {X.3.4.last-(X.3.4.length)+1} {B} {C} {4-4} {3.4}
% \Test:nnnnn {X.3.5.last-(X.3.5.length)+1} {B} {C} {4-5} {3.5}
% \end{test}
%
% \begin{function}{\Beanover}
%   \begin{syntax}
%     \cs{Beanover} \marg{key--value list}
%   \end{syntax}
%   The keys are the slide range specifiers.
% We do not accept key only items, they are managed by \cs{@@_error:n}.
% \meta{key--value} items are parsed by \cs{@@_parse:nn}.
% A group is open.
%    \begin{macrocode}
\NewDocumentCommand \Beanover { m } {
  \keyval_parse:NNn \@@_error:n \@@_parse:nn { #1 }
  \ignorespaces
}
%    \end{macrocode}
% If we use this command in the frame body, it will be executed for each
% different frame.
% If we use the frame option \texttt{beanover} instead, the command is executed
% only once, at the cost of a more verbose code.
%    \begin{macrocode}
\define@key{beamerframe}{beanover}{\Beanover{#1}}
%    \end{macrocode}
% \end{function}
% \subsubsection{Scanning named overlay specifications}
% Patch some beamer command to support |?(...)| instructions in overlay specifications.
% \begin{function}{\beamer@masterdecode}
% \begin{syntax}
% \cs{beamer@masterdecode} \marg{overlay specification}
% \end{syntax}
% Preprocess \meta{overlay specification} before \pkg{beamer} uses it.
% \begin{variable}{\l_ans_tl}
%   Storage for the translated overlay specification, where |?(...)| instructions are replaced by their static counterparts.
% \end{variable}
% \end{function}
% Save the original macro \cs{beamer@masterdecode} and then override it
% to properly preprocess the argument.
%    \begin{macrocode}
\cs_set_eq:NN \@@_beamer@masterdecode \beamer@masterdecode
\cs_set:Npn \beamer@masterdecode #1 {
  \group_begin:
  \tl_clear:N \l_ans_tl
  \@@_scan:nNN { #1 } \@@_eval:nN \l_ans_tl
  \exp_args:NNV
  \group_end:
  \@@_beamer@masterdecode \l_ans_tl
}
%    \end{macrocode}
% \begin{function}{\@@_scan:nNN}
% \begin{syntax}
% \cs{@@_scan:nNN} \marg{named overlay expression} \meta{eval} \meta{tl variable}
% \end{syntax}
% Scan the \meta{named overlay expression} argument and feed the \meta{tl variable} replacing |?(...)| instructions by their static counterpart
% with help from the \meta{eval} function, which is \cs{@@_eval:nN}.
% A group is created to use local variables:
% \begin{description}
%   \item[\cs{l_ans_tl}:] is the token list that will be appended to \meta{tl variable} on return.
% \end{description}
% \begin{variable}{\l_depth_int}
% Store the depth level in parenthesis grouping used when finding
%     the proper closing parenthesis balancing the opening parenthesis that
%     follows immediately a question mark in a \texttt{?(...)} instruction.
% \end{variable}
% \begin{variable}{\l_query_tl}
% Storage for the overlay query expression to be evaluated.
% \end{variable}
% \begin{variable}{\l_token_seq}
% The \meta{overlay expression} is split into the sequence of its tokens.
% \end{variable}
% \begin{variable}{\l_ask_bool}
% Whether a loop may continue.
% Controls the continuation of the main loop that scans
%     the tokens of the \meta{named overlay expression} looking for a question mark.
% \end{variable}
% \begin{variable}{\l_query_bool}
% Whether a loop may continue. 
% Controls the continuation of the secondary loop that scans
%     the tokens of the \meta{overlay expression} looking for an opening
%     parenthesis follow the question mark.
%     It then controls the loop looking for the balanced closing parenthesis.
% \end{variable}
% \begin{variable}{\l_token_tl}
% Storage for just one token.
% \end{variable}
% \end{function}
%    \begin{macrocode}
\cs_new:Npn \@@_scan:nNN #1 #2 #3 {
  \group_begin:
  \tl_clear:N \l_ans_tl
  \int_zero:N \l_depth_int
  \seq_clear:N \l_token_seq
%    \end{macrocode}
% Explode the \meta{named overlay expression} into a list of tokens:
%    \begin{macrocode}
  \regex_split:nnN {} { #1 } \l_token_seq
%    \end{macrocode}
% Run the top level loop to scan for a `|?|':
%    \begin{macrocode}
  \bool_set_true:N  \l_ask_bool
  \bool_while_do:Nn \l_ask_bool {
    \seq_pop_left:NN \l_token_seq \l_token_tl
    \quark_if_no_value:NTF \l_token_tl {
%    \end{macrocode}
% We reached the end of the sequence (and the token list),
% we end the loop here.
%    \begin{macrocode}
      \bool_set_false:N \l_ask_bool
    } {
%    \end{macrocode}
% \cs{l_token_tl} contains a `normal' token.
%    \begin{macrocode}
      \tl_if_eq:NnTF \l_token_tl { ? } {
%    \end{macrocode}
% We found a `|?|', we first gobble tokens until the next `|(|', ^^A ---)
% whatever they may be. In general, no tokens should be silently ignored.
%    \begin{macrocode}
        \bool_set_true:N \l_query_bool
        \bool_while_do:Nn \l_query_bool {
%    \end{macrocode}
% Get next token.
%    \begin{macrocode}
          \seq_pop_left:NN \l_token_seq \l_token_tl
          \quark_if_no_value:NTF \l_token_tl {
%    \end{macrocode}
% No opening parenthesis found, raise.
%    \begin{macrocode}
            \msg_fatal:nnx { beanover } { :n } {Missing~'('%---)
              ~after~a~?:~#1}
          } {
            \tl_if_eq:NnT \l_token_tl { ( %)
            } {
%    \end{macrocode}
% We found the `|(|` after the `|?|'. ^^A---)
% Increment the parenthesis depth to 1 (on first passage).
%    \begin{macrocode}
              \int_incr:N \l_depth_int
%    \end{macrocode}
% Record the forthcomming content in the \cs{l_query_tl} variable, ^^A(---
% up to the next balancing `|)|`.
%    \begin{macrocode}
              \tl_clear:N \l_query_tl
              \bool_while_do:Nn \l_query_bool {
%    \end{macrocode}
% Get next token.
%    \begin{macrocode}
                \seq_pop_left:NN \l_token_seq \l_token_tl
                \quark_if_no_value:NTF \l_token_tl {
%    \end{macrocode}
% We reached the end of the sequence and the token list with no closing ^^A(---
% `|)|'.
% We raise and end both bool while loops.
% As recovery we feed \cs{l_query_tl} with the missing ^^A(---
% `|)|'.
% |\l_depth_int| is 0 whenever |\l_query_bool| is false.
%    \begin{macrocode}
                  \msg_error:nnx { beanover } { :n } {Missing~%(---
                    `)':~#1 }
                  \int_do_while:nNnn \l_depth_int > 1 {
                    \int_decr:N \l_depth_int
                    \tl_put_right:Nn \l_query_tl {%(---
                    )}
                  }
                  \int_zero:N \l_depth_int
                  \bool_set_false:N \l_query_bool
                  \bool_set_false:N \l_ask_bool
                } {
                  \tl_if_eq:NnTF \l_token_tl { ( %---)
                  } {
%    \end{macrocode}
% We found a `|(|', ^^A---)
% increment the depth and append the token to \cs{l_query_tl}.
%    \begin{macrocode}
                    \int_incr:N \l_depth_int
                    \tl_put_right:NV \l_query_tl \l_token_tl
                  } {
%    \end{macrocode}
% This is not a `|(|'.^^A---)
%    \begin{macrocode}
                    \tl_if_eq:NnTF \l_token_tl { %(
                      )
                    } {
%    \end{macrocode}
% We found a  ^^A(---
% `|)|', decrement the depth.
%    \begin{macrocode}
                      \int_decr:N \l_depth_int
                      \int_compare:nNnTF \l_depth_int = 0 {
%    \end{macrocode}
% The depth level has reached 0: we found our balancing parenthesis
% of the |?(...)| instruction.
% We can append the evaluated slide ranges token list to \cs{l_ans_tl}
% and stop the inner loop.
%    \begin{macrocode}
  \exp_args:NV #2 \l_query_tl \l_ans_tl
  \bool_set_false:N \l_query_bool
                      } {
%    \end{macrocode}
% The depth has not yet reached level 0. ^^A(---
% We append the `|)|' to \cs{l_query_tl} because it is not the
% end of sequence marker.
%    \begin{macrocode}
                        \tl_put_right:NV \l_query_tl \l_token_tl
                      }
%    \end{macrocode}
% Above ends the code for a positive depth.
%    \begin{macrocode}
                    } {
%    \end{macrocode}
% The scanned token is not a `|(|' nor a `|)|',
% we append it as is to \cs{l_query_tl}.
%    \begin{macrocode}
                      \tl_put_right:NV \l_query_tl \l_token_tl
                    }
                  }
                }
%    \end{macrocode}
% Above ends the code for Not a  `|(|'^^A---)
%    \begin{macrocode}
              }
            }
%    \end{macrocode}
% Above ends the code for: Found the `|(|' after the `|?|' ^^A---)
%    \begin{macrocode}
          }
%    \end{macrocode}
% Above ends the code for not a no value quark.
%    \begin{macrocode}
        }
%    \end{macrocode}
% Above ends the code for the bool while loop to find the `|(|' after the `|?|'.^^A---)
%
% If we reached the end of the token list, then end both the current loop and its containing loop.
%    \begin{macrocode}
        \quark_if_no_value:NT \l_token_tl {
          \bool_set_false:N \l_query_bool
          \bool_set_false:N \l_ask_bool
        }
      } {
%    \end{macrocode}
% This is not a `|?|', append the token to right of \cs{l_ans_tl} and continue.
%    \begin{macrocode}
        \tl_put_right:NV \l_ans_tl \l_token_tl
      }
%    \end{macrocode}
% Above ends the code for the bool while loop to find a `|(|' after the `|?|' ^^A---)
%    \begin{macrocode}
    }
  }
%    \end{macrocode}
% Above ends the outer bool while loop to find `|?|' characters.
% We can append our result to \meta{tl variable}
%    \begin{macrocode}
  \exp_args:NNNV
  \group_end:
  \tl_put_right:Nn #3 \l_ans_tl
}
%    \end{macrocode}
% \begin{test}
% \cs_set:Npn \eval:nN #1 #2 {
%   \tl_put_right:Nn #2 { !(#1) }
% }
% \tl_clear:N \l_a_tl
% \__beanover_scan:nNN {1+1} \eval:nN \l_a_tl
% \__beanover_scan:nNN {?(A)} \eval:nN \l_a_tl
% \cs_set:Npn \Test:nnn #1 #2 #3 {
%   \tl_clear:N \l_a_tl
%   \__beanover_scan:nNN { #1 } \eval:nN \l_a_tl
%   \exp_args:NV
%   \__beanover_assert_equal:nnn \l_a_tl { #2 } {
%     Test~\cs{__beanover_scan:nNN}~#3
%   }
% }
% \Test:nnn {} {} {1}
% \Test:nnn {1+1} {1+1} {2}
% \Test:nnn {1?(+)1} {1!(+)1} {3}
% \Test:nnn {1?(+((X)(X))-)1} {1!(+((X)(X))-)1} {4}
% \end{test}
% Each new frame has its own set of slide ranges,
% we clear the property list on entering a new frame environment.
% Frame environments nested into other frame environments are not supported.
%    \begin{macrocode}
\AddToHook
  { env/beamer@framepauses/before }
  { \prop_gclear:N \g_@@_prop }
%    \end{macrocode}
%
% \subsubsection{Evaluation bricks}
%
%
% \begin{function}[pTF]{ \@@_if_first:nN }
% \begin{syntax}
% \cs{@@_if_first:nNTF} \marg{name} \meta{tl variable} \marg{true code} \marg{false code}
% \end{syntax}
% Append the first of the \meta{name} slide range
% to the \meta{tl variable}.
% Cache the result.
% Execute \meta{true code} when there is a \meta{first}, \meta{false code} otherwise.
% \end{function}
%    \begin{macrocode}
\prg_new_conditional:Npnn \@@_if_first:nN #1 #2 { p, T, F, TF } {
  \@@_if_in:nTF { #1//A } {
    \tl_put_right:Nx #2 { \@@_item:n { #1//A } }
    \prg_return_true:
  } {
    \group_begin:
    \tl_clear:N \l_ans_tl
    \@@_if_in:nTF { #1/A } {
      \@@_eval:xN {
        \@@_item:n { #1/A }
      } \l_ans_tl
    } {
      \bool_if:nTF {
        \@@_if_in_p:n { #1/L } && \@@_if_in_p:n { #1/Z }
      } {
        \@@_eval:xN {
          \@@_item:n { #1/Z } - ( \@@_item:n { #1/L } - 1 )
        } \l_ans_tl
      } {
        \@@_if_in:nT { #1/C } {
          \bool_set_true:N \l_no_counter_bool
          \@@_eval:xN {
            \@@_item:n { #1/C }
          } \l_ans_tl
        }
      }
    }
    \tl_if_empty:NTF \l_ans_tl {
      \group_end:
      \prg_return_false:
    } {
      \@@_gput:nV { #1//A } \l_ans_tl
      \exp_args:NNNV
      \group_end:
      \tl_put_right:Nn #2 \l_ans_tl
      \prg_return_true:
    }
  }
}
%    \end{macrocode}
%
% \begin{function}{\@@_first:nN, \@@_first:VN}
% \begin{syntax}
% \cs{@@_first:nN} \marg{name} \meta{tl variable}
% \end{syntax}
% Append the start of the \meta{name} slide range
% to the \meta{tl variable}.
% Cache the result.
% \end{function}
%    \begin{macrocode}
\cs_new:Npn \@@_first:nN #1 #2 {
  \@@_if_first:nNF { #1 } #2 {
    \msg_error:nnn { beanover } { :n } { Range~with~no~first:~#1 }
  }
}
\cs_generate_variant:Nn \@@_first:nN { VN }
%    \end{macrocode}
% \begin{test}
% \cs_set:Npn \__beanover_eval:xN #1 #2 { \tl_set:Nx #2 { #1 } }
% \cs_set:Npn \Test:nnn #1 #2 #3 {
%   \Beanover { X = #1 }
%   \tl_clear:N \l_ans_tl
%   \__beanover_first:nN { X } \l_ans_tl
%   \__beanover_assert_equal:xxn { \l_ans_tl } { #2 }
%     { Test~\cs{__beanover_first:nN }~#3 }
% }
% \Test:nnn { 123 } { 123 } { 1 }
% \Test:nnn { 123 } { 123 } { 2 }
% \Test:nnn { :A::C } { C-(A-1) } { 3 }
% \Test:nnn { :A::C } { C-(A-1) } { 4 }
% \Test:nnn { ::C:A } { C-(A-1) } { 5 }
% \Test:nnn { ::C:A } { C-(A-1) } { 6 }
% \end{test}
% \begin{function}[pTF]{\@@_if_length:nN}
% \begin{syntax}
% \cs{@@_length_p:nN} \marg{name} \meta{tl variable}
% \cs{@@_if_length:nNTF} \marg{name} \meta{tl variable} \marg{true code} \marg{false code}
% \end{syntax}
% Append the length of the \meta{name} slide range to \meta{tl variable}
% Execute \meta{true code} when there is a \meta{length}, \meta{false code} otherwise.
% \end{function}
%    \begin{macrocode}
\prg_new_conditional:Npnn \@@_if_length:nN #1 #2 { p, T, F, TF } {
  \@@_if_in:nTF { #1//L } {
    \tl_put_right:Nx #2 { \@@_item:n { #1//L } }
    \prg_return_true:
  } {
    \group_begin:
    \tl_clear:N \l_ans_tl
    \@@_if_in:nTF { #1/L } {
      \@@_eval:xN {
        \@@_item:n { #1/L }
      } \l_ans_tl
    } {
      \bool_if:nT {
        \@@_if_in_p:n { #1/A } && \@@_if_in_p:n { #1/Z }
      } {
        \@@_eval:xN {
          \@@_item:n { #1/Z } - (\@@_item:n { #1/A } - 1)
        } \l_ans_tl
      }
    }
    \tl_if_empty:NTF \l_ans_tl {
      \group_end:
      \prg_return_false:
    } {
      \@@_gput:nV { #1//L } \l_ans_tl
      \exp_args:NNNV
      \group_end:
      \tl_put_right:Nn #2 \l_ans_tl
      \prg_return_true:
    }
  }
}
%    \end{macrocode}
% \begin{function}{\@@_length:nN, \@@_length:VN}
% \begin{syntax}
% \cs{@@_length:nN} \marg{name} \meta{tl variable}
% \end{syntax}
% Append the length of the \meta{name} slide range to \meta{tl variable}
% \begin{test}
% \cs_set:Npn \__beanover_eval:xN #1 #2 { \tl_set:Nx #2 { #1 } }
% \cs_set:Npn \Test:nnn #1 #2 #3 {
%   \Beanover { X = #1 }
%   \tl_clear:N \l_ans_tl
%   \__beanover_length:nN { X } \l_ans_tl
%   \__beanover_assert_equal:xxn { \l_ans_tl } { #2 }
%     { Test~\cs{__beanover_length:nN }~#3 }
%   \tl_clear:N \l_ans_tl
%   \__beanover_length:nN { X } \l_ans_tl
%   \__beanover_assert_equal:xxn { \l_ans_tl } { #2 }
%     { Test~\cs{__beanover_length:nN }~#3 }
% }
% \Test:nnn { 123:421 } { 421 } { 1 }
% \Test:nnn { A::C  } { C-(A-1) } { 2 }
% \Test:nnn { :B::C } { B } { 3 }
% \Test:nnn { ::C:B } { B } { 4 }
% \end{test}
% \end{function}
%    \begin{macrocode}
\cs_new:Npn \@@_length:nN #1 #2 {
  \@@_if_length:nNF { #1 } #2 {
    \msg_error:nnn { beanover } { :n } { Range~with~no~length:~#1 }
  }
}
\cs_generate_variant:Nn \@@_length:nN { VN }
%    \end{macrocode}
% \begin{test}
% \cs_set:Npn \__beanover_eval:xN #1 #2 { \tl_set:Nx #2 { #1 } }
% \cs_set:Npn \Test:nnn #1 #2 #3 {
%   \Beanover { X = #1 }
%   \tl_clear:N \l_ans_tl
%   \__beanover_length:nN { X } \l_ans_tl
%   \__beanover_assert_equal:xxn { \l_ans_tl } { #2 }
%     { Test~\cs{__beanover_length:nN }~#3 }
% }
% \Test:nnn { A:B } { B } { 1 }
% \Test:nnn { A:B } { B } { 2 }
% \Test:nnn { A::B } { B-(A-1) } { 3 }
% \Test:nnn { A::B } { B-(A-1) } { 4 }
% \Test:nnn { :B::C } { B } { 5 }
% \Test:nnn { :B::C } { B } { 6 }
% \Test:nnn { ::C:B } { B } { 7 }
% \Test:nnn { ::C:B } { B } { 8 }
% \end{test}
%
% \begin{function}[pTF]{\@@_if_last:nN}
% \begin{syntax}
% \cs{@@_if_last_p:nN} \marg{name} \meta{tl variable}
% \cs{@@_if_last:nNTF} \marg{name} \meta{tl variable} \marg{true code} \marg{false code}
% \end{syntax}
% \end{function}
%    \begin{macrocode}
\prg_new_conditional:Npnn \@@_if_last:nN #1 #2 { p, T, F, TF } {
  \@@_if_in:nTF { #1//Z } {
    \tl_put_right:Nx #2 { \@@_item:n { #1//Z } }
    \prg_return_true:
  }  {
    \group_begin:
    \tl_clear:N \l_ans_tl
    \@@_if_in:nTF { #1/Z } {
      \@@_eval:xN {
        \@@_item:n { #1/Z }
      } \l_ans_tl
    } {
      \@@_get:nNT { #1/A } \l_a_tl {
        \@@_get:nNT { #1/L } \l_b_tl {
          \@@_eval:xN {
            \l_a_tl + \l_b_tl - 1
          } \l_ans_tl
        }
      }
    }
    \tl_if_empty:NTF \l_ans_tl {
      \group_end:
      \prg_return_false:
    } {
      \@@_gput:nV { #1//Z } \l_ans_tl
      \exp_args:NNNV
      \group_end:
      \tl_put_right:Nn #2 \l_ans_tl
      \prg_return_true:
    }
  }
}
%    \end{macrocode}
% \begin{function}{\@@_last:nN, \@@_last:VN}
% \begin{syntax}
% \cs{@@_last:nN} \marg{name} \meta{tl variable}
% \end{syntax}
% Append the last index of the \meta{name} slide range to \meta{tl variable}
% \end{function}
%    \begin{macrocode}
\cs_new:Npn \@@_last:nN #1 #2 {
  \@@_if_last:nNF { #1 } #2 {
    \msg_error:nnn { beanover } { :n } { Range~with~no~last:~#1 }
  }
}
\cs_generate_variant:Nn \@@_last:nN { VN }
%    \end{macrocode}
% \begin{test}
% \cs_set:Npn \__beanover_eval:xN #1 #2 { \tl_set:Nx #2 { #1 } }
% \cs_set:Npn \Test:nnn #1 #2 #3 {
%   \Beanover { X = #1 }
%   \tl_clear:N \l_ans_tl
%   \__beanover_last:nN { X } \l_ans_tl
%   \__beanover_assert_equal:xxn { \l_ans_tl } { #2 }
%     { Test~\cs{__beanover_last:nN }~#3-a }
%   \tl_clear:N \l_ans_tl
%   \__beanover_if_last:nNTF { X } \l_ans_tl {
%     \tl_set:Nn \l_a_tl { SUCCESS }
%   } {
%     \tl_set:Nn \l_a_tl { FAILURE }
%   }
%   \__beanover_assert_equal:xxn { \l_a_tl } { SUCCESS }
%     { Test~\cs{__beanover_last:nN }~#3-b }
%   \__beanover_assert_equal:xxn { \l_ans_tl } { #2 }
%     { Test~\cs{__beanover_last:nN }~#3-c }
%   \tl_clear:N \l_ans_tl
%   \__beanover_if_last:nNTF { Y } \l_ans_tl {
%     \tl_set:Nn \l_a_tl { FAILURE }
%   } {
%     \tl_set:Nn \l_a_tl { SUCCESS }
%   }
%   \__beanover_assert_equal:xxn { \l_a_tl } { SUCCESS }
%     { Test~\cs{__beanover_last:nN }~#3-d }
%   \tl_clear:N \l_ans_tl
%   \__beanover_last:nN { X } \l_ans_tl
%   \__beanover_assert_equal:xxn { \l_ans_tl } { #2 }
%     { Test~\cs{__beanover_last:nN }~#3-a }
%   \tl_clear:N \l_ans_tl
%   \__beanover_if_last:nNTF { X } \l_ans_tl {
%     \tl_set:Nn \l_a_tl { SUCCESS }
%   } {
%     \tl_set:Nn \l_a_tl { FAILURE }
%   }
%   \__beanover_assert_equal:xxn { \l_a_tl } { SUCCESS }
%     { Test~\cs{__beanover_last:nN }~#3-b }
%   \__beanover_assert_equal:xxn { \l_ans_tl } { #2 }
%     { Test~\cs{__beanover_last:nN }~#3-c }
%   \tl_clear:N \l_ans_tl
%   \__beanover_if_last:nNTF { Y } \l_ans_tl {
%     \tl_set:Nn \l_a_tl { FAILURE }
%   } {
%     \tl_set:Nn \l_a_tl { SUCCESS }
%   }
%   \__beanover_assert_equal:xxn { \l_a_tl } { SUCCESS }
%     { Test~\cs{__beanover_last:nN }~#3-d }
% }
% \Test:nnn { ::C } { C } { 1 }
% \Test:nnn { ::C } { C } { 2 }
% \Test:nnn { A::C } { C } { 3 }
% \Test:nnn { A::C } { C } { 4 }
% \Test:nnn { A:B } { A+B-1 } { 5 }
% \Test:nnn { A:B } { A+B-1 } { 6 }
% \end{test}
% \begin{function}[pTF]{\@@_if_next:nN}
% \begin{syntax}
% \cs{@@_if_next_p:nN} \marg{name} \meta{tl variable}
% \cs{@@_if_next:nNTF} \marg{name} \meta{tl variable} \marg{true code} \marg{false code}
% \end{syntax}
% Append the index after the \meta{name} slide range to the \meta{tl variable}.
% Execute \meta{true code} when there is a \meta{next} index, \meta{false code} otherwise.
% \end{function}
%    \begin{macrocode}
\prg_new_conditional:Npnn \@@_if_next:nN #1 #2 { p, T, F, TF } {
  \@@_if_in:nTF { #1//N } {
    \tl_put_right:Nx #2 { \@@_item:n { #1//N } }
    \prg_return_true:
  } {
    \group_begin:
    \@@_get:nNTF { #1/Z } \l_ans_tl {
      \tl_put_right:Nn \l_ans_tl { +1 }
    } {
      \@@_get:nNT { #1/A } \l_a_tl {
        \@@_get:nNT { #1/L } \l_b_tl {
          \@@_eval:xN {
            \l_a_tl + \l_b_tl
          } \l_ans_tl
        }
      }
    }
    \tl_if_empty:NTF \l_ans_tl {
      \group_end:
      \prg_return_false:
    } {
      \@@_gput:nV { #1//N } \l_ans_tl
      \exp_args:NNNV
      \group_end:
      \tl_put_right:Nn #2 \l_ans_tl
      \prg_return_true:
    }
  }
}
%    \end{macrocode}
% \begin{function}{\@@_next:nN, \@@_next:VN}
% \begin{syntax}
% \cs{@@_next:nN} \marg{name} \meta{tl variable}
% \end{syntax}
% Append the index after the \meta{name} slide range to the \meta{tl variable}.
% \end{function}
%    \begin{macrocode}
\cs_new:Npn \@@_next:nN #1 #2 {
  \@@_if_next:nNF { #1 } #2 {
    \msg_error:nnn { beanover } { :n } { Range~with~no~next:~#1 }
  }
}
\cs_generate_variant:Nn \@@_next:nN { VN }
%    \end{macrocode}
% \begin{test}
% \cs_set:Npn \__beanover_eval:xN #1 #2 { \tl_set:Nx #2 { #1 } }
% \cs_set:Npn \Test:nnn #1 #2 #3 {
%   \Beanover { X = #1 }
%   \tl_clear:N \l_ans_tl
%   \__beanover_next:nN { X } \l_ans_tl
%   \__beanover_assert_equal:xxn { \l_ans_tl } { #2 }
%     { Test~\cs{__beanover_next:nN }~#3-a }
%   \tl_clear:N \l_ans_tl
%   \__beanover_if_next:nNTF { X } \l_ans_tl {
%     \tl_set:Nn \l_a_tl { SUCCESS }
%   } {
%     \tl_set:Nn \l_a_tl { FAILURE }
%   }
%   \__beanover_assert_equal:xxn { \l_a_tl } { SUCCESS }
%     { Test~\cs{__beanover_next:nN }~#3-b }
%   \__beanover_assert_equal:xxn { \l_ans_tl } { #2 }
%     { Test~\cs{__beanover_next:nN }~#3-c }
%   \tl_clear:N \l_ans_tl
%   \__beanover_if_next:nNTF { Y } \l_ans_tl {
%     \tl_set:Nn \l_a_tl { FAILURE }
%   } {
%     \tl_set:Nn \l_a_tl { SUCCESS }
%   }
%   \__beanover_assert_equal:xxn { \l_a_tl } { SUCCESS }
%     { Test~\cs{__beanover_next:nN }~#3-d }
%   \tl_clear:N \l_ans_tl
%   \__beanover_next:nN { X } \l_ans_tl
%   \__beanover_assert_equal:xxn { \l_ans_tl } { #2 }
%     { Test~\cs{__beanover_next:nN }~#3-a }
%   \tl_clear:N \l_ans_tl
%   \__beanover_if_next:nNTF { X } \l_ans_tl {
%     \tl_set:Nn \l_a_tl { SUCCESS }
%   } {
%     \tl_set:Nn \l_a_tl { FAILURE }
%   }
%   \__beanover_assert_equal:xxn { \l_a_tl } { SUCCESS }
%     { Test~\cs{__beanover_next:nN }~#3-b }
%   \__beanover_assert_equal:xxn { \l_ans_tl } { #2 }
%     { Test~\cs{__beanover_next:nN }~#3-c }
%   \tl_clear:N \l_ans_tl
%   \__beanover_if_next:nNTF { Y } \l_ans_tl {
%     \tl_set:Nn \l_a_tl { FAILURE }
%   } {
%     \tl_set:Nn \l_a_tl { SUCCESS }
%   }
%   \__beanover_assert_equal:xxn { \l_a_tl } { SUCCESS }
%     { Test~\cs{__beanover_next:nN }~#3-d }
% }
% \Test:nnn { ::C } { C+1 } { 1 }
% \Test:nnn { A::C } { C+1 } { 2 }
% \Test:nnn { A:B } { A+B } { 3 }
% \Test:nnn { :B::C } { C+1 } { 4 }
% \end{test}
% \begin{function}{\@@_free_counter:nN, \@@_free_counter:VN}
% \begin{syntax}
% \cs{@@_free_counter:nN} \marg{name} \meta{tl variable}
% \end{syntax}
% Append the value of the counter associated to the \marg{name} slide range
% to the right of \meta{tl variable}.
% There is no branching variant because, we always return some value,
% `|1|' by default.
% \end{function}
%    \begin{macrocode}
\cs_new:Npn \@@_free_counter:nN #1 #2 {
  \group_begin:
  \tl_clear:N \l_ans_tl
  \@@_get:nNF { #1/C } \l_ans_tl {
    \@@_if_first:nNF { #1 } \l_ans_tl {
      \@@_if_last:nNF { #1 } \l_ans_tl {
        \tl_set:Nn \l_ans_tl { 1 }
      }
    }
  }
  \@@_gput:nV { #1/C } \l_ans_tl
  \exp_args:NNNV
  \group_end:
  \tl_put_right:Nn #2 \l_ans_tl
}
\cs_generate_variant:Nn \@@_free_counter:nN { VN }
%    \end{macrocode}
% \begin{test}
% \cs_set:Npn \__beanover_eval:xN #1 #2 { \tl_set:Nx #2 { #1 } }
% \cs_set:Npn \Test:nnn #1 #2 #3 {
%   \Beanover { X = #1 }
%   \tl_clear:N \l_ans_tl
%   \__beanover_free_counter:nN { X } \l_ans_tl
%   \__beanover_assert_equal:xxn { \l_ans_tl } { #2 }
%     { Test~\cs{__beanover_free_counter:nN }~#3 }
%   \__beanover_gclear:
% }
% \Test:nnn { 1 } { 1 } { 1 }
% \Test:nnn { 123 } { 123 } { 2 }
% \Test:nnn { ::123 } { 123 } { 3 }
% \Test:nnn { A+B } { A+B } { 4 }
% \Test:nnn { ::A+B } { A+B } { 5 }
% \end{test}
% \begin{function}{\@@_counter:nN, \@@_counter:VN}
% \begin{syntax}
% \cs{@@_counter:nN} \marg{name} \meta{tl variable}
% \end{syntax}
% Append the value of the counter associated to the \marg{name} slide range
% to the right of \meta{tl variable}.
% The value always lays in between the range, whenever possible.
% \end{function}
%    \begin{macrocode}
\cs_new:Npn \@@_counter:nN #1 #2 {
  \group_begin:
  \@@_free_counter:nN { #1 } \l_ans_tl
%    \end{macrocode}
% If there is a \meta{first}, use it to bound the result from below.
%    \begin{macrocode}
  \tl_clear:N \l_a_tl
  \@@_if_first:nNT { #1 } \l_a_tl {
    \fp_compare:nNnT { \l_ans_tl } < { \l_a_tl } {
      \tl_set:NV \l_ans_tl \l_a_tl
    }
  }
%    \end{macrocode}
% If there is a \meta{last}, use it to bound the result from above.
%    \begin{macrocode}
  \tl_clear:N \l_a_tl
  \@@_if_last:nNT { #1 } \l_a_tl {
    \fp_compare:nNnT { \l_ans_tl } > { \l_a_tl } {
      \tl_set:NV \l_ans_tl \l_a_tl
    }
  }
  \exp_args:NNNx
  \group_end:
  \tl_set:Nn #2 { \fp_eval:n { round(\l_ans_tl) } }
}
\cs_generate_variant:Nn \@@_counter:nN { VN }
%    \end{macrocode}
% \begin{test}
% \cs_set:Npn \__beanover_eval:xN #1 #2 { \tl_set:Nx #2 { #1 } }
% \cs_set:Npn \Test:nnn #1 #2 #3 {
%   \__beanover_parse:nn { X } { #1 }
%   \tl_clear:N \l_ans_tl
%   \__beanover_counter:nN { X } \l_ans_tl
%   \__beanover_assert_equal:xxn { \l_ans_tl } { #2 }
%     { Test~\cs{__beanover_counter:nN }~#3 }
%   \__beanover_gclear:
% }
% \Test:nnn { 1 } { 1 } { 1 }
% \Test:nnn { 123 } { 123 } { 2 }
% \Test:nnn { ::123 } { 123 } { 3 }
% \end{test}
% \begin{function}{\@@_index:nnN, \@@_index:VVN}
% \begin{syntax}
% \cs{@@_index:nnN} \marg{name} \marg{integer path} \meta{tl variable}
% \end{syntax}
% Append the value of the counter associated to the \marg{name} slide range
% to the right of \meta{tl variable}.
% The value always lays in between the range, whenever possible.
% \end{function}
%    \begin{macrocode}
\cs_new:Npn \@@_index:nnN #1 #2 #3 {
  \group_begin:
  \tl_set:Nn \l_name_tl { #1 }
  \regex_split:nnNTF { \. } { #2 } \l_split_seq {
    \seq_pop_left:NN \l_split_seq \l_a_tl
    \seq_pop_right:NN \l_split_seq \l_a_tl
    \seq_map_inline:Nn \l_split_seq {
      \tl_set_eq:NN \l_b_tl \l_name_tl
      \tl_put_right:Nn \l_b_tl { . ##1 }
      \exp_args:Nx     
      \@@_get:nN { \l_b_tl / A } \l_c_tl
      \quark_if_no_value:NTF \l_c_tl {
        \tl_set_eq:NN \l_name_tl \l_b_tl
      } {
        \tl_set_eq:NN \l_name_tl \l_c_tl      
      }
    }
  } {
\msg_error:nnx { beanover } { :n } { Internal~error (#1/#2) }
  }
  \tl_clear:N \l_b_tl
  \exp_args:Nx
  \@@_get:nN { \l_name_tl.\l_a_tl / A } \l_b_tl
  \quark_if_no_value:NTF \l_b_tl {
    \exp_args:NV
    \@@_first:nN \l_name_tl \l_ans_tl
    \tl_put_right:Nx \l_ans_tl { + \l_a_tl - 1}
  } {
    \tl_set_eq:NN \l_ans_tl \l_b_tl
  }
  \exp_args:NNNx
  \group_end:
  \tl_set:Nn #3 { \fp_eval:n { round(\l_ans_tl) } }
}
%    \end{macrocode}
% \begin{test}
% \cs_set:Npn \__beanover_eval:xN #1 #2 { \tl_set:Nx #2 { #1 } }
% \cs_set:Npn \Test:nnn #1 #2 #3 #4 #5 {
%   \Beanover { #1 }
%   \tl_clear:N \l_ans_tl
%   \__beanover_index:nnN { #2 } { #3 } \l_ans_tl
%   \__beanover_assert_equal:xxn { \l_ans_tl } { #4 }
%     { Test~\cs{__beanover_index:nnN }~#5 }
%   \__beanover_gclear:
% }
% \Test:nnn { A = 1 } { A } { .1 } { 1 } { 1 }
% \Test:nnn { A = 1 } { A } { .2 } { 2 } { 2 }
% \Test:nnn { A = 1, B.2 = A } { B } { .2.3 } { 3 } { 3 }
% \Test:nnn { A = 2, B.2 = A } { B } { .2.4 } { 5 } { 4 }
% \Test:nnn { A = 2, B.2.3 = A } { B } { .2.3.4 } { 5 } { 5 }
% \Test:nnn { A = 2, B.2.3 = A, A.4 = 25 } { B } { .2.3.4 } { 25 } { 6 }
% \end{test}

% \begin{function}{\@@_incr:nn, \@@_incr:nnN, \@@_incr:VnN, \@@_incr:VVN}
% \begin{syntax}
% \cs{@@_incr:nn}  \marg{name} \marg{offset}
% \cs{@@_incr:nnN} \marg{name} \marg{offset} \meta{tl variable}
% \end{syntax}
% \end{function}
% Increment the free counter position accordingly.
% When requested, put the result in the \meta{tl variable}.
% The result will lay within the declared range.
%    \begin{macrocode}
\cs_new:Npn \@@_incr:nn #1 #2 {
  \group_begin:
  \tl_clear:N \l_a_tl
  \@@_free_counter:nN { #1 }  \l_a_tl
  \tl_clear:N \l_ans_tl
  \@@_eval:xN { \l_a_tl + ( #2 ) } \l_ans_tl
  \@@_gput:nV { #1/C } \l_ans_tl
  \group_end:
}
\cs_new:Npn \@@_incr:nnN #1 #2 #3 {
  \@@_incr:nn { #1 } { #2 }
  \@@_counter:nN { #1 } #3
}
\cs_generate_variant:Nn \@@_incr:nnN { VnN }
\cs_generate_variant:Nn \@@_incr:nnN { VVN }
%    \end{macrocode}
% \begin{test}
% \cs_set:Npn \__beanover_eval:xN #1 #2 {
%   \tl_set:Nx #2 {
%     \fp_eval:n { round( #1 ) }
%   }
% }
% \cs_set:Npn \Test:nnn #1 #2 #3 {
%   \tl_clear:N \l_ans_tl
%   \__beanover_free_counter:nN { X } \l_ans_tl
%   \__beanover_assert_equal:xxn { \l_ans_tl } { #1 }
%     { Test~\cs{__beanover_incr:nN }~#3-a }
%   \tl_clear:N \l_ans_tl
%   \__beanover_counter:nN { X } \l_ans_tl
%   \__beanover_assert_equal:xxn { \l_ans_tl } { #2 }
%     { Test~\cs{__beanover_incr:nN }~#3-b }
%   \__beanover_gclear:
% }
% \Beanover { X = 123 }
% \Test:nnn { 123 } { 123 } { 1 }
% \Beanover { X = 123 }
% \__beanover_incr:nn { X } { -100 }
% \Test:nnn { 23 } { 123 } { 2 }
% \Beanover { X = ::123 }
% \Test:nnn { 123 } { 123 } { 3 }
% \Beanover { X = ::123 }
% \__beanover_incr:nn { X } { 100 }
% \Test:nnn { 223 } { 123 } { 4 }
% \end{test}
%
% \subsubsection{Evaluation}
%
% \begin{function}{\@@_resolve:nnN, \@@_resolve:VVN}
% \begin{syntax}
% \cs{@@_resolve:nnN} \marg{name} \marg{path} \meta{tl variable}
% \end{syntax}
% Resolve the \meta{name} and \meta{path} into a key that is put into the \meta{tl variable}.
% \end{function}
%    \begin{macrocode}
\cs_new:Npn \@@_resolve:nnN #1 #2 #3 {
  \group_begin:
  \tl_set:Nn \l_a_tl { #1 }
  \regex_split:nnNT { \. } { #2 } \l_split_seq {
    \seq_pop_left:NN \l_split_seq \l_b_tl
    \seq_map_inline:Nn \l_split_seq {
      \tl_set_eq:NN \l_b_tl \l_a_tl
      \tl_put_right:Nn \l_b_tl { . ##1 }
      \exp_args:Nx     
      \@@_get:nN { \l_b_tl / A } \l_c_tl
      \quark_if_no_value:NTF \l_c_tl {
        \tl_set_eq:NN \l_a_tl \l_b_tl
      } {
        \tl_set_eq:NN \l_a_tl \l_c_tl      
      }
    }
  }
  \exp_args:NNNV
  \group_end:
  \tl_set:Nn #3 \l_a_tl
}
\cs_generate_variant:Nn \@@_resolve:nnN { VVN }
%    \end{macrocode}
% \begin{test}
% \cs_set:Npn \__beanover_eval:xN #1 #2 {
%   \tl_set:Nx #2 { #1 }
% }
% \cs_set:Npn \Test:nnn #1 #2 #3 #4 #5 {
%   \tl_clear:N \l_ans_tl
%   \Beanover { #1 }
%   \__beanover_resolve:nnN { #2 } { #3 } \l_ans_tl
%   \__beanover_assert_equal:xxn { \l_ans_tl } { #4 }
%     { Test~\cs{__beanover_resolve:nnN }~#5 }
%   \__beanover_gclear:
% }
% \Test:nnn {} { X } { .1.2 } { X.1.2 } { 1 }
% \__beanover_parse:nn { X.1 } { Y }
% \__beanover_get:nN { X.1 / A } \l_ans_tl
% \__beanover_assert_equal:xxn { \l_ans_tl } { Y }
%   { Test~\cs{__beanover_resolve:nnN }~A }
% \Test:nnn { X.1 = Y } { X } { .1.2 } { Y.2 } { 2 }
% \Test:nnn { X.1 = Y, Y.2 = Z } { X } { .1.2 } { Z } { 3 }
% \Test:nnn { X.1.2 = Z } { X } { .1.2 } { Z } { 3 }
% \end{test}
%
% \begin{function}{\@@_append:nN, \@@_append:VN}
% \begin{syntax}
% \cs{@@_append:nN} \marg{integer expression} \meta{tl variable}
% \end{syntax}
% Evaluates the \meta{integer expression},
% replacing all the named specifications by their static
% counterpart then put the result to
% the right of the \meta{tl variable}.
% Executed within a group.
% Heavily used by \cs{@@_eval_query:nN}, where \meta{integer expression}
% was enclosed in `|?(...)|'.
% Local variables: 
% \begin{variable}{\l_ans_tl}
% For the content of \meta{tl variable}
% \end{variable}
% \begin{variable}{\l_split_seq}
% The sequence of queries and non queries.
% \end{variable}
% \begin{variable}{\l_split_int}
%  Is the index of the non queries, before all the catched groups.
% \end{variable}
% \begin{variable}{\l_name_tl}
%    Storage for \cs{l_split_seq} items that represent names.
% \end{variable}
% \begin{variable}{\l_path_tl}
%    Storage for \cs{l_split_seq} items that represent paths.
% \end{variable}
% \begin{variable}{\l@@_static_tl}
%    Storage for the static values of named slide lists.
% \end{variable}
% \begin{variable}{\l_group_tl}
%    Storage for capture groups.
% \end{variable}
% \end{function}
%    \begin{macrocode}
\cs_new:Npn \@@_append:nN #1 #2 {
  \group_begin:
%    \end{macrocode}
% Local variables:
%    \begin{macrocode}
  \tl_clear:N  \l_ans_tl
  \int_zero:N  \l_split_int
  \seq_clear:N \l_split_seq
  \tl_clear:N  \l_name_tl
  \tl_clear:N  \l_path_tl
  \tl_clear:N  \l_group_tl
  \tl_clear:N  \l_a_tl
%    \end{macrocode}
% Implementation:
%    \begin{macrocode}  
  \regex_split:NnN \c@@_split_regex { #1 } \l_split_seq
  \int_set:Nn \l_split_int { 1 }
  \tl_set:Nx \l_ans_tl { \seq_item:Nn \l_split_seq { \l_split_int } }
%    \end{macrocode}
% \begin{function}{\switch:nTF}
% \begin{syntax}
% \cs{switch:nTF} \marg{capture group number} \marg{black code} \marg{white code}
% \end{syntax}
% Helper function to locally set the |\l_group_tl| variable
% to the captured group \meta{capture group number}
% and branch.
% \end{function}
%    \begin{macrocode}
  \cs_set:Npn \switch:nNTF ##1 ##2 ##3 ##4 {
    \tl_set:Nx ##2 {
      \seq_item:Nn \l_split_seq { \l_split_int + ##1 }
    }
    \tl_if_empty:NTF ##2 { ##4 } { ##3 }
  }
%    \end{macrocode}
% Main loop.
%    \begin{macrocode}
  \int_while_do:nNnn { \l_split_int } < { \seq_count:N \l_split_seq } {
    \switch:nNTF 1 \l_name_tl {
%    \end{macrocode}
% \begin{myList}
% \item Case \texttt{++\meta{name}\meta{integer path}.n}.
%    \begin{macrocode}
      \switch:nNTF 2 \l_path_tl {
        \@@_resolve:VVN \l_name_tl \l_path_tl \l_name_tl
      } { }
      \@@_incr:VnN \l_name_tl 1 \l_ans_tl
    } {
      \switch:nNTF 3 \l_name_tl {
%    \end{macrocode}
% \begin{test}
% \cs_set:Npn \__beanover_eval:xN #1 #2 {
%   \tl_set:Nx #2 { #1 }
% }
% \cs_set:Npn \Test:nnn #1 #2 #3 #4 {
%   \tl_clear:N \l_ans_tl
%   \Beanover { #1 }
%   \__beanover_append:nN { #2 } \l_ans_tl
%   \__beanover_assert_equal:xxn { \l_ans_tl } { #3 }
%     { Test~\cs{__beanover_resolve:nnN }~#4 }
% }
% \Test:nnn {X=123} { ++X.n } { 124 } { 1 }
% \end{test}
% \item Cases \texttt{\meta{name}\meta{integer path}...}.
%    \begin{macrocode}
        \tl_set:Nn \l_b_tl {
          \switch:nNTF 4 \l_path_tl {
            \@@_resolve:VVN \l_name_tl \l_path_tl \l_name_tl
          } { }
        }
        \switch:nNTF 5 \l_a_tl {
%    \end{macrocode}
% \item Case \texttt{...length}.
%    \begin{macrocode}
          \l_b_tl
          \@@_length:VN \l_name_tl \l_ans_tl
        } {
          \switch:nNTF 6 \l_a_tl {
%    \end{macrocode}
% \item Case \texttt{...last}.
%    \begin{macrocode}
            \l_b_tl
            \@@_last:VN \l_name_tl \l_ans_tl
          } {
            \switch:nNTF 7 \l_a_tl {
%    \end{macrocode}
% \item Case \texttt{...next}.
%    \begin{macrocode}
              \l_b_tl
              \@@_next:VN \l_name_tl \l_ans_tl
            } {
              \switch:nNTF 8 \l_a_tl {
%    \end{macrocode}
% \item Case \texttt{...range}.
%    \begin{macrocode}
                \l_b_tl
                \@@_range:VN \l_name_tl \l_ans_tl
              } {
                \switch:nNTF 9 \l_a_tl {
%    \end{macrocode}
% \item Case \texttt{...n}.
%    \begin{macrocode}
                  \l_b_tl
                  \switch:nNTF { 10 } \l_a_tl {
%    \end{macrocode}
% \item Case \texttt{...+=\meta{integer}}.
%    \begin{macrocode}
                    \@@_incr:VVN \l_name_tl \l_a_tl \l_ans_tl
                  } {
                    \@@_counter:VN \l_name_tl \l_ans_tl
                  }
                } {
                  \switch:nNTF 4 \l_path_tl {
                    \exp_args:NVV
                    \@@_counter:nnN \l_name_tl \l_path_tl \l_ans_tl
                  } {
                    \exp_args:NV
                    \@@_counter:nnN \l_name_tl { .1 } \l_ans_tl
                  }
                }
              }
            }
          }
        }
      }
    }
    \int_add:Nn \l_split_int { 11 }
    \tl_put_right:Nx \l_ans_tl { \seq_item:Nn \l_split_seq { \l_split_int } }
  }
  \exp_args:NNNx
  \group_end:
  \tl_put_right:Nn #2 { \fp_to_int:n { \l_ans_tl } }
}
\cs_generate_variant:Nn \@@_append:nN { VN }
%    \end{macrocode}
% \end{myList}
%
% \begin{function}{\@@_eval_query:nN}
% \begin{syntax}
% \cs{@@_eval_query:Nn} \marg{overlay query} \meta{seq variable}
% \end{syntax}
% Evaluates the single \meta{overlay query}, 
% which is expected to contain no comma.
% Extract a range specification from the argument,
% replaces all the named overlay specifications by their static counterparts,
% make the computation then append the result to 
% the right of the \meta{seq variable}.
% Ranges are supported with the colon syntax.
% This is executed within a local group.
% Below are local variables and constants.
% \begin{variable}{\l_a_tl}
% Storage for the first index of a range.
% \end{variable}
% \begin{variable}{\l_b_tl}
% Storage for the last index of a range, or its length.
% \end{variable}
% \begin{variable}{\c@@_A_cln_Z_regex}
% Used to parse slide range overlay specifications.
% Next are the capture groups.
% \end{variable}
%    \begin{macrocode}
\regex_const:Nn \c@@_A_cln_Z_regex {
  \A \s* (?:
%    \end{macrocode}
% \begin{myList}
% \item 2: \meta{first}
%    \begin{macrocode}
      ( [^:]* ) \s* :
%    \end{macrocode}
% \item 3: second optional colon
%    \begin{macrocode}
      (:)? \s*
%    \end{macrocode}
% \item 4: \meta{length}
%    \begin{macrocode}
      ( [^:]* )
%    \end{macrocode}
% \item 5: standalone \meta{first}
%    \begin{macrocode}
    | ( [^:]+ )
  ) \s* \Z
}
%    \end{macrocode}
% \end{myList}
% \begin{test}
% \__beanover_test_extract:Nnnn \c__beanover_A_cln_Z_regex
%   {A:B} {A:B,{A},{},{B},{}} {Test~1}
% \__beanover_test_extract:Nnnn \c__beanover_A_cln_Z_regex
% {A} {A,{},{},{},A} {Test~2}
% \__beanover_test_extract:Nnnn \c__beanover_A_cln_Z_regex
%   {A:B} {A:B,{A},{},{B},{}} {Test~1}
% \__beanover_test_extract:Nnnn \c__beanover_A_cln_Z_regex
%   {A::B} {A::B,{A},{:},{B},{}} {Test~1}
% \__beanover_test_extract:Nnnn \c__beanover_A_cln_Z_regex
%   {A:} {A:,{A},{},{},{}} {Test~1}
% \__beanover_test_extract:Nnnn \c__beanover_A_cln_Z_regex
%   {A::} {A::,{A},{:},{},{}} {Test~1}
% \__beanover_test_extract:Nnnn \c__beanover_A_cln_Z_regex
%   {::B} {::B,{},:,B,{}} {Test~1}
% \__beanover_test_extract:Nnnn \c__beanover_A_cln_Z_regex
%   {::} {::,{},:,{},{}} {Test~1}
% \__beanover_test_extract:Nnnn \c__beanover_A_cln_Z_regex
%   {:} {:,{},{},{},{}} {Test~1}
% \end{test}
%
%    \begin{macrocode}
\cs_new:Npn \@@_eval_query:nN #1 #2 {
  \regex_extract_once:NnNTF \c@@_A_cln_Z_regex {
    #1
  } \l_match_seq {
    \tl_clear:N \l_ans_tl
    \bool_set_false:N \l_no_counter_bool
    \bool_set_true:N  \l_no_range_bool
%    \end{macrocode}
% \end{function}
% \begin{function}{\switch:nNTF}
% \begin{syntax}
% \cs{switch:nNTF} \marg{capture group number} \meta{tl variable} \marg{black code} \marg{white code}
% \end{syntax}
% Helper function to locally set the \meta{tl variable}
% to the captured group \meta{capture group number}
% and branch depending on the emptyness of this variable.
% \end{function}
%    \begin{macrocode}
    \cs_set:Npn \switch:nNTF ##1 ##2 ##3 ##4 {
      \tl_set:Nx ##2 {
        \seq_item:Nn \l_split_seq { ##1 }
      }
      \tl_if_empty:NTF ##2 { ##4 } { ##3 }
    }
    \switch:nNTF 5 \l_a_tl {
%    \end{macrocode}
% \emoji{left-speech-bubble} Single expression
%    \begin{macrocode}
      \bool_set_false:N \l_no_range_bool
      \@@_append:VN \l_a_tl \l_ans_tl
      \seq_put_right:NV #1 \l_ans_tl
    } {
      \switch:nNTF 2 \l_a_tl {
        \switch:nNTF 4 \l_b_tl {
          \switch:nNTF 3 \l_a_tl {
%    \end{macrocode}
% \emoji{left-speech-bubble} \texttt{\meta{first}::\meta{last}} range
%    \begin{macrocode}
            \@@_append:VN \l_a_tl \l_ans_tl
            \tl_put_right:Nn \l_ans_tl { - }
            \@@_append:VN \l_b_tl \l_ans_tl
            \seq_put_right:NV #1 \l_ans_tl
          } {
%    \end{macrocode}
% \emoji{left-speech-bubble} \texttt{\meta{first}:\meta{length}} range
%    \begin{macrocode}
            \@@_append:VN \l_a_tl \l_ans_tl       
            \tl_put_right:Nx \l_ans_tl { - }        
            \tl_put_right:Nx \l_a_tl { - ( \l_b_tl ) + 1}  
            \@@_append:VN \l_b_tl \l_ans_tl
            \seq_put_right:NV #1 \l_ans_tl
          }
        } {
%    \end{macrocode}
% \emoji{left-speech-bubble} \texttt{\meta{first}:} and \texttt{\meta{first}::} range
%    \begin{macrocode}
          \@@_append:VN \l_a_tl \l_ans_tl
          \tl_put_right:Nn \l_ans_tl { - }
          \seq_put_right:NV #1 \l_ans_tl
        }
      } {
        \switch:nNTF 4 \l_b_tl {
          \switch:nNTF 3 \l_a_tl {
%    \end{macrocode}
% \emoji{left-speech-bubble} \texttt{::\meta{last}} range
%    \begin{macrocode}
            \tl_put_right:Nn \l_ans_tl { - }
            \@@_append:VN \l_a_tl \l_ans_tl
            \seq_put_right:NV #1 \l_ans_tl            
          } {
\msg_error:nnx { beanover } { :n } { Syntax~error(Missing~first):~#1 }
          }
        } {
%    \end{macrocode}
% \emoji{left-speech-bubble} \texttt{:} or \texttt{::} range
%    \begin{macrocode}
          \seq_put_right:Nn #2 { - }
        } 
      }
    }
  } {
%    \end{macrocode}
% Error
%    \begin{macrocode}
    \msg_error:nnn { beanover } { :n } { Syntax~error:~#1 }
  }
}
%    \end{macrocode}
%
% \begin{function}{\@@_eval:nN}
% \begin{syntax}
% \cs{@@_eval:nN} \marg{overlay query list} \meta{tl variable}
% \end{syntax}
% Evaluates the comma separated list of \meta{overlay query}'s,
% replacing all the named overlay specifications and integer expressions
% by their static counterparts by calling \cs{@@_eval_query:nN},
% then append the result to the right of the \meta{tl variable}.
% This is executed within a local group.
% Below are local variables and constants used
% throughout the body of this function.
% \begin{variable}{\l_query_seq}
% Storage for a sequence of \meta{query}'s obtained by splitting a comma separated list.
% \end{variable}
% \begin{variable}{\l_ans_seq}
% Storage of the evaluated result.
% \end{variable}
% \begin{variable}{\c@@_comma_regex}
% Used to parse slide range overlay specifications.
%    \begin{macrocode}
\regex_const:Nn \c@@_comma_regex { \s* , \s* }
%    \end{macrocode}
% \end{variable}
% No other variable is used.
% \end{function}
%    \begin{macrocode}
\cs_new:Npn \@@_eval:nN #1 #2 {
  \group_begin:
%    \end{macrocode}
% Local variables declaration
%    \begin{macrocode}
  \tl_clear:N  \l_a_tl
  \tl_clear:N  \l_b_tl
  \tl_clear:N  \l_ans_tl
  \seq_clear:N \l_ans_seq
  \seq_clear:N \l_query_seq
%    \end{macrocode}
% In this main evaluation step, we evaluate the integer expression and put
% the result in a variable which content will be copied after the group is closed.
% We authorize comma separated expressions
% and \texttt{\meta{first}::\meta{last}} range expressions as well.
% We first split the expression around commas, into \cs{l_query_seq}.
%    \begin{macrocode}
  \@@_append:nN { #1 } \l_ans_tl
  \exp_args:NNV
  \regex_split:NnN \c@@_comma_regex \l_ans_tl \l_query_seq
%    \end{macrocode}
% Then each component is evaluated and the result is stored in \cs{l_seq}
% that we must clear before use.
%    \begin{macrocode}
  \seq_map_tokens:Nn \l_query_seq {
    \@@_eval_query:Nn \l_ans_seq
  }
%    \end{macrocode}
% We have managed all the comma separated components, we collect them back
% and append them to \meta{tl variable}.
%    \begin{macrocode}
  \exp_args:NNNx
  \group_end:
  \tl_put_right:Nn #2 { \seq_use:Nn \l_ans_seq , }
}
\cs_generate_variant:Nn \@@_eval:nN { VN, xN }
%    \end{macrocode}
%
% \begin{function}{\BeanoverEval}
%   \begin{syntax}
%     \cs{BeanoverEval} \oarg{tl variable} \marg{overlay queries}
%   \end{syntax}
% \meta{overlay queries} is the argument of |?(...)| instructions.
% This is a comma separated list of single \meta{overlay query}'s.
%
% This function evaluates the \meta{overlay queries} and store the result in the \meta{tl variable} when provided
% or leave the result in the input stream. Forwards to \cs{@@_eval:nN}
% within a group. \cs{l_ans_tl} is used locally to store the result.
% \end{function}
%    \begin{macrocode}
\NewExpandableDocumentCommand \BeanoverEval { s o m } {
  \group_begin:
  \tl_clear:N \l_ans_tl
  \IfBooleanTF { #1 } {
    \bool_set_true:N  \l_no_counter_bool
  } {
    \bool_set_false:N \l_no_counter_bool
  }
  \@@_eval:nN { #3 } \l_ans_tl
  \IfValueTF { #2 } {
    \exp_args:NNNV
    \group_end:
    \tl_set:Nn #2 \l_ans_tl
  } {
    \exp_args:NV
    \group_end: \l_ans_tl
  }
}
%    \end{macrocode}
%
%
% \subsubsection{Reseting slide ranges}
% \begin{function}{\BeanoverReset}
%   \begin{syntax}
%     \cs{BeanoverReset} \oarg{first value} \marg{Slide list name}
%   \end{syntax}
%    \begin{macrocode}
\NewDocumentCommand \BeanoverReset { O{1} m } {
  \@@_reset:nn { #1 } { #2 }
  \ignorespaces
}
%    \end{macrocode}
% Forwards to \cs{@@_reset:nn}.
% \end{function}
%
% \begin{function}{\@@_reset:nn}
% \begin{syntax}
% \cs{@@_reset:nn} \marg{first value} \marg{slide list name}
% \end{syntax}
% Reset the counter to the given \meta{first value}.
% Clean the cached values also (not usefull).
% \end{function}
%    \begin{macrocode}
\cs_new:Npn \@@_reset:nn #1 #2 {
  \bool_if:nTF {
    \@@_if_in_p:n { #2/A } || \@@_if_in_p:n { #2/Z }
  } {
    \@@_gremove:n { #2/C }
    \@@_gremove:n { #2//A }
    \@@_gremove:n { #2//L }
    \@@_gremove:n { #2//Z }
    \@@_gremove:n { #2//N }
    \@@_gput:nn { #2/C0 } { #1 }
  } {
    \msg_warning:nnn { beanover } { :n } { Unknown~name:~#2 }
  }
}
%    \end{macrocode}
%
%    \begin{macrocode}
\makeatother
\ExplSyntaxOff
%    \end{macrocode}
% \end{implementation}
%
%    \begin{macrocode}
%</package>
%    \end{macrocode}
%
% \StopEventually{\setlength\IndexMin{200pt}  \PrintIndex }
%
%\Finale
%
