% !TeX encoding = UTF-8
% !TeX program = lualatex
% !TeX proot = ...

\documentclass{beamer}

\let \BeanovesSticky \Sticky

\RequirePackage[absolute,overlay]{textpos}
\TPGrid[\baselineskip,\baselineskip]{16}{16}

\RequirePackage{array}
\RequirePackage{calc}
\RequirePackage{emoji}
\RequirePackage{graphicx}
\newcommand\myBulb{%
  \smash{\raisebox{-0.15\baselineskip}{\large\emoji{bulb}}}%
}
\newcommand\myWatch{%
  \smash{\raisebox{-0.15\baselineskip}{\large\emoji{hourglass}}}%
}

\RequirePackage{tcolorbox}
\tcbuselibrary{skins}
\tcbset{
  myCodeBox/.style= {
    colframe=gray,
    sharpish corners,
    boxrule=2pt,
    skin=enhanced,
    drop fuzzy shadow=lightgray,
    boxsep=0.25\baselineskip,
    left=0mm,
    right=0mm,
    top=0mm,
    bottom=0mm,
  },
  myShortCode/.style= {
    colframe=gray,
    sharpish corners,
    boxrule=2pt,
    skin=enhanced,
    drop fuzzy shadow=lightgray,
    boxsep=0mm,
    left=0.25\baselineskip,
    right=0.25\baselineskip,
    top=0.2\baselineskip,
    bottom=0.0\baselineskip,
    nobeforeafter,
    baseline=2mm,
  },
}
\NewDocumentCommand\myShortCodeBox{O{}mm} {%
  {%
    \ttfamily\small
    \setlength{\tabcolsep}{0mm}%
    \tcbox[myShortCode,#1]{\begin{tabular}{#2}#3\end{tabular}}%
  }%
}
\NewDocumentEnvironment {myCodeBox} { O{} m } {%
  \begin{tcolorbox}[myCodeBox,#1]%
  \begin{myLongCode}{#2}%
} {%
  \end{myLongCode}%
  \end{tcolorbox}
}
\NewDocumentEnvironment {myLongCode} { m } {%
  \bgroup
  \setlength{\tabcolsep}{0mm}\ttfamily\small
  \renewcommand\arraystretch{1}%
  \begin{tabular} {#1}
} {%
  \end{tabular}
  \egroup
}

\NewDocumentCommand \myEmph { m } {%
  \textit{\bfseries #1}%
}

\RequirePackage{tikz}
\usetikzlibrary{
  arrows.meta,
  calc,
  math,
  shapes.geometric,
}
\RequirePackage{sticky}
\RequirePackage{fontenc}
\RequirePackage{unicode-math}
\RequirePackage{verbatimbox}
\RequirePackage{catchfile}
\RequirePackage{beanoves-debug}

\RequirePackage{color}
\definecolor{MyGreen}{RGB}{0,111,0}

\RequirePackage{transparent}
\newenvironment{myTransparentenv}{\only{\transparent{0.3}}}{}

\RequirePackage{graphicx}

\newcommand{\tlap}[1]{\mbox{\vbox to 0pt{\vss\hbox{#1}}}}
\newcommand{\tblap}[1]{\mbox{\vbox to 0pt{\vss\hbox{#1}\vss}}}
\newcommand{\blap}[1]{\mbox{\vbox to 0pt{\hbox{#1}\vss}}}

\setmonofont[Mapping=tex-text,Scale=0.85]{Menlo}

\ExplSyntaxOn
\NewDocumentCommand \myRoundCornerImg { O{} O{} m } {
  \group_begin:
    \tl_if_empty:nTF { #2 } {%
      \hbox_set:Nn \g_tmpa_box {%
        \includegraphics { #3 }%
      }
    } {%
      \hbox_set:Nn \g_tmpa_box {%
        \includegraphics [ #2 ] { #3 }%
      }
    }
    \tikzmath {
      \w = \dim_eval:n { \box_wd:N \g_tmpa_box };
      \h = \dim_eval:n { \box_ht:N \g_tmpa_box };
      \c = \dim_eval:n { 0.05 \box_wd:N \g_tmpa_box };
    }
    \tikz [
      x=\w pt,
      y=\h pt,
    ] {
      \clip [rounded~corners=\c pt] (-0.5,-0.5) rectangle (0.5,0.5); 
      \node [inner~sep=0pt,#1] at (0,0) {\box_use:N \g_tmpa_box};
    }
    \group_end:
}

\NewDocumentCommand\myImgSticky { O{} r() O{} m } {
  \group_begin:
    \tl_if_empty:nTF { #3 } {
      \hbox_set:Nn \g_tmpa_box {
        \includegraphics { #4 }
      }
    } {
      \hbox_set:Nn \g_tmpa_box {
        \includegraphics [ #3 ] { #4 }
      }
    }
    \tikzmath {
      \w = \dim_eval:n { \box_wd:N \g_tmpa_box };
      \h = \dim_eval:n { \box_ht:N \g_tmpa_box };
      \c = \dim_eval:n { 0.05 \box_wd:N \g_tmpa_box };
    }
    \hbox_set:Nn \g_tmpb_box {
      \tikz [
        x=\w pt,
        y=\h pt,
      ] {
        \clip [rounded~corners=\c pt] (-0.5,-0.5) rectangle (0.5,0.5); 
        \node [inner~sep=0pt] at (0,0) {\box_use:N \g_tmpa_box};
      }
    }
    \tikz [
      overlay,
      remember~picture,
    ] {
      \node [inner~sep=0pt,#1] at (#2) {\box_use:N \g_tmpb_box};
    }
  \group_end:
  \ignorespaces
}
\ExplSyntaxOff

%%%%%%%%%%%%%%%%%%%%%%%%%%%%%
\newsavebox{\myBoxAW}


\tikzset {
  my code/.style = {
    inner sep=0.5\baselineskip,
    fill=white,
    draw=gray,
    line width=2pt,
    font=\ttfamily\small,
  },
  my arrow/.style= {
    -{Stealth},
    MyGreen,
    thick,
    densely dotted,
  },
}

\newcommand\myCodek{\rlap{\raisebox{-0.5mm}{\scriptsize k}}\phantom{x}}
\NewDocumentCommand\myInlineCode {sm} {%
  \texttt{\bfseries\IfBooleanT{#1}{\color{MyGreen}}#2}%
}
%\newcommand\myMeta[1]{\phantom{<}\llap{$⟨$}\textit{#1}\rlap{$⟩$}\phantom{>}}
\ExplSyntaxOn
\newcommand\myMeta[1]{%
  \group_begin:
  \hbox_set:Nn \l_tmpa_box {x}
  \skip_horizontal:n {0.5 \box_wd:N \l_tmpa_box}
  \clap{$⟨$}
  \skip_horizontal:n {0.5 \box_wd:N \l_tmpa_box}
  \textit{#1}
  %\hbox_set:Nn \l_tmpa_box {>}
  \skip_horizontal:n {0.5 \box_wd:N \l_tmpa_box}
  \clap{$⟩$}
  \skip_horizontal:n {0.5 \box_wd:N \l_tmpa_box}
  \group_end:
}
\ExplSyntaxOff

\CatchFileDef{\useMyStepII}{frames/step2}{}
\CatchFileDef{\useMyStepIII}{frames/step3}{}
\CatchFileDef{\useMyChronologyII}{frames/chronology2}{}
\CatchFileDef{\useMyChronologyIII}{frames/chronology3}{}
\CatchFileDef{\useMyPractice}{frames/practice}

\begin{document}

%\transfade[duration=0.5]
\transdissolve[duration=0.5]

\CatchFileDef{\useMyIntroduction}{frames/introduction}{}
\begin{frame}
{Beanoves demonstration manual}
{Introduction}
\useMyIntroduction
\end{frame}

\CatchFileDef{\useMyFacts}{frames/2-beamer facts}{}
\begin{frame}
{Beanoves demonstration manual}
{Beamer facts}
\transpush[direction=180]
\useMyFacts
\end{frame}


%
% STEP 1
%
%\CatchFileDef{\useMyStepI}{frames/step1}{}
%\begin{frame}
%{Bealover example about animals: Simple items}
%{Slide \insertslidenumber}
%\transpush[direction=180]
%\useMyStepI
%\end{frame}
%%
% STEP 2
%
%\begin{frame}
%{Bealover example: Uncovered items}
%{Slide \insertslidenumber}
%\Beanoves {
%  Sticky1 = 5,
%  Sticky2 = Sticky1.next,
%  Sticky3 = Sticky2.next : 3,
%  Sticky4 = Sticky3.next : 3,
%}
%\transpush[direction=180]
%\useMyStepII
%\end{frame}
%
% Step 2 chronology
%
\Beanoves {
  Sticky1 = 1 : 7,
}
%\begin{frame}
%{Bealover example: Uncovered items}
%{\large Chronology}
%%\transpush[direction=180]
%\useMyChronologyII
%\end{frame}
%
% Step 3
%
\Beanoves {
  Air      = 1 :: Gannet.last,
  Chameleo = Air.2:1,
  Gannet   = Chameleo.next:1,
  Water   = Air.next :: Picasso.last,
  Octopus = Water.2:1,
  Starfish = Octopus.next:1,
  StickyStarfish = Starfish.1,
  Picasso = Starfish.next,
  PicassoTrans = 1 :: Picasso.0,
  Summary  = Water.next,
}
\begin{frame}
{Bealover example: Uncovered items + images}
{Slide \insertslidenumber}
%\transpush[direction=180]
\useMyStepIII
\end{frame}
\end{document}
%
%
% Step 3 chronology
%
\Beanoves {
  Sticky1 = 1 : 12,
  Sticky2 = Sticky1.next : 5,
  Code = Sticky2.next : CodeSummary.last - Code.0,
  CodeAir = Code.2,
  CodeWater = CodeAir.next,
  CodeSummary = CodeWater.next,
  Usage = Code.next,
}
\begin{frame}
{Beanoves example: Uncovered items + images}
{\large Chronology}
\transpush[direction=180]
\useMyChronologyIII
\end{frame}

\begin{frame}
{Bealover example: Uncovered items + images}
{\large Step back}
\transpush[direction=270]
\Beanoves {
  Air      = 1:0,
  Chameleo = 1,
  Gannet   = 1,
  Water   = 1:2,
  Octopus = 1,
  Starfish = 2,
  StickyStarfish = 1:0,
  Picasso = 2:0,
  PicassoTrans = 1 : 2,
  Summary  = 1:0,
}
\only<1-2> {
  \useMyStepIII
}
\vbox to0pt{\hbox{%
\only<2> {
  \tikz [
    remember picture,
    overlay,
  ] {
    \node [my code,yshift=-2.5\baselineskip] at (current page.center) [text width=0.75\textwidth,anchor=north] {%
\backslash only < \textcolor{MyGreen}{\bfseries?(Water.2)} - > \{\\
\phantom{xx}\backslash only < \textcolor{MyGreen}{\bfseries?(Water.3)} - > \{\backslash transparent\{0.3\}\}\\
\phantom{xx}\% code to display the Octopus...\\
\}\\
    }%
  }
}
}}%
\BeanovesSticky<1>(current page.south){0.8\textwidth}[anchor=south,yshift=1\baselineskip]{\bfseries%
• On next slide, the "Octopus" image is expected to be transparent\\
• Let us see the source code
}%
\BeanovesSticky<2>(current page.south){0.85\textwidth}[anchor=south,yshift=1\baselineskip]{\bfseries%
• \textcolor{MyGreen}{\texttt{\bfseries?(Water.2)}} is replaced by \textcolor{MyGreen}{\bfseries\texttt{5}}
and \textcolor{MyGreen}{\texttt{\bfseries?(Water.3)}} by \textcolor{MyGreen}{\bfseries\texttt{6}}\\
• The octopus is displayed from slide 5 and transparent from slide 6
}%
\end{frame}
%
\Beanoves {
  Command = 1 : 2,
  Example = Command.next : 2,
}
\begin{frame}
{Bealover manual}
{\large Named overlay specification}
\begin{myCodeBox}{l}%
\backslash begin \{frame\}\\
\backslash Bealover \{\\
\phantom{xx}%
\alt<?(Example.range)>{%
  {\only<?(Example.1)>{\color{MyGreen}}Air \phantom{xx}= 1\phantom{xxxxxxxx}: 2,}
}{%
  \myMeta{range name\myCodek} =
  \myMeta{range start\myCodek}
  [: \myMeta{range length\myCodek}],
}\\
\phantom{xx}%
\alt<?(Example.range)>{%
{\only<?(Example.2)>{\color{MyGreen}}Water = Air.next : 3,}
}{%
...
}\\
\}
\end{myCodeBox}
\begin{itemize}
\item \myMeta{range name\myCodek} is like a variable name
\item \myMeta{range start\myCodek} is an \emph{\bfseries extended integer expression},
\item \myMeta{range length\myCodek} is an \emph{\bfseries extended integer expression},
\item when not provided, \myMeta{range duration\myCodek}  defaults to 1
\end{itemize}
\vspace{6\baselineskip}
\BeanovesSticky<?(Example.1)>(current page.south){0.8\textwidth}[anchor=south,yshift=2\baselineskip]{\bfseries%
Revisit the example:\\
• The Air topic is the first one: position is \texttt{\color{MyGreen}1}\\
• 2 images for the Air topic: duration is \texttt{\color{MyGreen}2}
}%
\BeanovesSticky<?(Example.2)>(current page.south){0.8\textwidth}[anchor=south,yshift=2\baselineskip]{\bfseries%
Revisit the example:\\
• The Water topic follows the Air topic: position is \texttt{\color{MyGreen}Air.next}\\
• 3 images for the Water topic: duration is \texttt{\color{MyGreen}3}
}%
\end{frame}
%
%
%
\begin{frame}
{Bealover manual}
{\large Defining logical overlay ranges}
\transpush[direction=180]
\Beanoves {
  Command = 1 : 2,
  Example = Command.next : 3,
}
\begin{myCodeBox}{l}%
\backslash begin \{frame\}\\
\backslash Bealover \{\\
\phantom{xx}%
\alt<?(Example.range)>{%
  {\only<?(Example.1)>{\color{MyGreen}}Air \phantom{xx}= 1\phantom{xxxxxxxx}: 2,}
}{%
  \myMeta{name\myCodek} = \myMeta{start\myCodek} : \myMeta{length\myCodek},
}\\
\alt<?(Example.range)>{%
\phantom{xx}%
{\only<?(Example.2)->{\color{MyGreen}}Water = Air.next : 3,}
}{%
\phantom{xx}...
}\\
\}
\alt<?(Example.range)>{\}}{}
\end{myCodeBox}
Range starts and lengths are arithmetical expression involving raw integers as well as next \emph{\bfseries aliases}.\\
\vspace{0.25\baselineskip}
\begin{myLongCode}{l>{\quad$⟷$\quad}l}
\hline
\textnormal{Alias} & \textnormal{Integer value}\\\hline
\only<?(Example.1)>{\color{MyGreen}Air.1}%
\only<?(Example.2)->{\color{MyGreen}Water.1}%
\only<?(Command.range)>{\myMeta{name\myCodek}.1}
&
\only<?(Example.1)>{ \color{MyGreen}1}%
\only<?(Example.2)->{ \color{MyGreen}3}%
\only<?(Command.range)>{\myMeta{start\myCodek}}
\\
\only<?(Example.1)>{\color{MyGreen}Air.2}%
\only<?(Example.2)->{\color{MyGreen}Water.2}%
\only<?(Command.range)>{\myMeta{name\myCodek}.2}
&
\only<?(Example.1)>{ \color{MyGreen}2}%
\only<?(Example.2)->{ \color{MyGreen}4}%
\only<?(Command.range)>{\myMeta{start\myCodek}+ 1}
\\
\only<?(Example.1)>{{\color{MyGreen}Air.}\myMeta{i}}%
\only<?(Example.2)->{{\color{MyGreen}Water.}\myMeta{i}}%
\only<?(Command.range)>{\myMeta{name\myCodek}.\myMeta{i}}
&
\only<?(Example.1)>{\myMeta{i}}%
\only<?(Example.2)->{\myMeta{i}+ 2}%
\only<?(Command.range)>{\myMeta{start\myCodek}+\myMeta{i}- 1}
\\
\only<?(Example.1)>{\rlap{\color{MyGreen}Air.length}}%
\only<?(Example.2)->{\rlap{\color{MyGreen}Water.length}}%
\visible<?(Command.range)>{\myMeta{name\myCodek}.length}
&
\only<?(Example.1)>{ \color{MyGreen}2}%
\only<?(Example.2)->{ \color{MyGreen}3}%
\only<?(Command.range)>{\myMeta{length\myCodek}}
\\
\only<?(Example.1)>{\color{MyGreen}Air.next}%
\only<?(Example.2)->{\color{MyGreen}Water.next}%
\only<?(Command.range)>{\myMeta{name\myCodek}.next}
&
\only<?(Example.1)>{ \color{MyGreen}3}%
\only<?(Example.2)->{ \color{MyGreen}6}%
\only<?(Command.range)>{\myMeta{start\myCodek}+\myMeta{length\myCodek}}
\\
\only<?(Example.1)>{\color{MyGreen}Air.last}%
\only<?(Example.2)->{\color{MyGreen}Water.last}%
\only<?(Command.range)>{\myMeta{name\myCodek}.last}
&
\only<?(Example.1)>{ \rlap{\color{MyGreen}2}}%
\only<?(Example.2)->{ \rlap{\color{MyGreen}5}}%
\visible<?(Command.range)>{\myMeta{start\myCodek}+\myMeta{length\myCodek}- 1}%
\\\hline
\end{myLongCode}
\vspace{1\baselineskip}
\BeanovesSticky<?(Example.0)-?(Example.last)>(current page.south east){0.35\textwidth}[anchor=south east,yshift=3\baselineskip,xshift=-1.5\baselineskip]{\bfseries%
Revisit the example...\\
\visible<?(Example.last)>{%
\myInlineCode*{Water.1 = Air.next}\\
\myInlineCode*{Air.last = Water.0}
}
}%
\end{frame}

\begin{frame}
{Bealover manual}
{\large Overlay specification query}
\Beanoves {
  Normal = 1 : 1,
  Extended = Normal.next : 2,
}
\begin{itemize}
\item Simple specifications\hfill
\myShortCodeBox[baseline=4mm]{l}{%
\backslash only < 4 > \phantom{xxxxxxxxx}\{...\}\\
\backslash only < 1 - 3 >\phantom{xxxxxx}\{...\}\\
}
\item Incremental specifications\hfill
\myShortCodeBox[baseline=4mm]{l}{%
\backslash only < + > \phantom{xxxxxxxxx}\{...\}\\
\backslash only < +(\textit{\bfseries<i>}) > \phantom{xxxx}\{...\}\\
}
\item \only<-?(Extended.0)>{\transparent{0.3}}\bfseries\color{MyGreen}Specification queries\hfill
\visible<?(Extended.range)>{%
\myShortCodeBox[baseline=2.5mm]{l}{%
\backslash only < {\color{MyGreen}?(\textit{\bfseries<query>})} > \{...\}\\
}\vphantom{\myShortCodeBox[baseline=4mm]{l}{%
x\\
x\\
}}%
}%
\end{itemize}
\vspace{1\baselineskip}
\visible<?(Extended.2)->{%
A query may be used in an overlay specification wherever an integer or a range can be.%
\\\myInlineCode{\backslash only} may be replaced by any specification aware command.
\vspace{1\baselineskip}

}%
\end{frame}
%
\begin{frame}
{Bealover manual}
{\large Overlay specification query syntax}
\Beanoves {
  Main = 1 : 2,
  Sticky = Main.next
}
\begin{itemize}
\item Position specifications
\begin{myCodeBox}{l}%
\only<?(Sticky)>{\color{MyGreen}\bfseries}%
?(\myMeta{integer expression with aliases})
\end{myCodeBox}
\item Explicit range specifications
\begin{myCodeBox}{l}%
\only<?(Sticky)>{\color{MyGreen}\bfseries}%
?(\myMeta{start expression} : {<length expression>})
\end{myCodeBox}
Both integer expressions accept aliases.
\item Logical range specifications with a \emph{\textbf{range alias}:}
\setlength{\tabcolsep}{0mm}
\\[0.25\baselineskip]
\begin{tabular}{>{\ttfamily}l>{\ttfamily\quad$⟷$\quad}l}
\myMeta{name\myCodek}.range
&
\myMeta{name\myCodek}.1 - \myMeta{name\myCodek}.last
\end{tabular}
\\[0.25\baselineskip]
where ``\texttt{-}'' stands for a dash and not a minus sign.
\begin{myCodeBox}{l}%
?(\myMeta{name\myCodek}.range)
\end{myCodeBox}
\end{itemize}
\visible<2->{%
Range queries and beamer ranges must not be combined like in \myInlineCode{?(Air.range)-10}, leading to the incorrect syntax \myInlineCode{1-2-10}.
}
\BeanovesSticky<?(Sticky)>(current page.south){0.75\textwidth}[anchor=south,yshift=1.5\baselineskip]{\bfseries%
\myBulb\ The middle slide of the Air topic is
\begin{center}\myInlineCode*{?((Air.1+Air.last)/2)}.\end{center}
\myBulb\ What corresponds to next query?
\begin{center}\myInlineCode*{?(Water.0 : Water.length + 2)}\end{center}
}%
\end{frame}
%
\begin{frame}
{Bealover manual}
{\large Incremental specifications}
\Beanoves {
  Main = 2 : 4,
  Example = 0:0,
  Sticky = Main.next,
}
\begin{myCodeBox}{l}%
\backslash begin \{frame\}\\
\backslash Bealover \{\\
\phantom{xx}%
\alt<?(Example.range)>{%
  {\only<?(Example.1)>{\color{MyGreen}}Air \phantom{xx}= 1\phantom{xxxxxxxx}: 2,}
}{%
  \myMeta{name\myCodek} = \myMeta{start\myCodek} : \myMeta{length\myCodek},
}\\
\alt<?(Example.range)>{%
\phantom{xx}%
{\only<?(Example.2)->{\color{MyGreen}}Water = Air.next : 3,}
}{%
\phantom{xx}...
}\\
\}
\end{myCodeBox}
Each logical overlay range has a current slide which number is \myEmph{cursor}, with dedicated alias and operations.
Within a specification query:
\begin{itemize}
\item
\only<-?(++Main-1)>{\transparent{0.3}}%
\myInlineCode*{\myMeta{name\myCodek}}, with not following ``\myInlineCode{.}'', is an alias for the \myEmph{cursor}
\item
\only<-?(++Main-1)>{\transparent{0.3}}%
\myInlineCode*{++\myMeta{name\myCodek}} stands for the \myEmph{cursor} once incremented by 1
\item
\only<-?(++Main-1)>{\transparent{0.3}}%
\myInlineCode*{\myMeta{name\myCodek}+=\myMeta{i}} stands for the \myEmph{cursor} once incremented by \myMeta{i}.
\item
\only<-?(++Main-1)>{\transparent{0.3}}%
\myInlineCode*{\myMeta{name\myCodek}.reset} stands for the \myEmph{cursor} once reset.
\end{itemize}
%
\BeanovesSticky<?(Sticky)>(current page.south){0.5\textwidth}[anchor=south,yshift=1.5\baselineskip]{\bfseries%
\myWatch\ Revisit the example...
}%
\vspace{2\baselineskip}
%
\end{frame}


%\begin{frame}
%{Bealover manual}
%{\large Incremental specifications in practice}
%\Beanoves {
%  X1 = 1 : W1.last - X1.0,
%  Sticky1a = X1.1:2,
%  Sticky1b = Sticky1a.next + 1: 2,
%  W1 = Sticky1a.next : 3,
%  X2 = X1.next : W2.last - X2.0,
%  Sticky2a = X2.1 : 2,
%  Sticky2b = Sticky2a.next + 1 : 2,
%  W2 = Sticky2a.next : 3,
%}
%\useMyPractice
%\end{frame}




%
\begin{frame}
{Bealover manual}
{\large Why aliases are helpful}
\Beanoves {
  Main = 1 : 2,
}
\begin{itemize}
\item
As soon as one leaves basic frame layouts to make presentations more attractive and efficient, then bealover aliases should come into play.
\item
One can organize the slides with logical names for a better understanding: aliases and integer expressions rather than raw integers make specifications more explicit
\item
Adding or removing a slide from one slide range does not significantly affect the other slide ranges.
\end{itemize}
\end{frame}
%
\end{document}
