% !TeX encoding = UTF-8
% !TeX program = lualatex
% !TeX proot = ...

\documentclass{beamer}

\RequirePackage{demo}

\CatchFileDef{\useMyIntroduction}{frames/introduction}{}
\CatchFileDef{\useMyFacts}{frames/2-beamer facts}{}
\CatchFileDef{\useMyStepI}{frames/step1}{}
\CatchFileDef{\useMyStepII}{frames/step2}{}
\CatchFileDef{\useMyChronologyII}{frames/chronology2}{}
\CatchFileDef{\useMyStepIII}{frames/step3}{}
\CatchFileDef{\useMyChronologyIII}{frames/chronology3}{}
\CatchFileDef{\useMyPractice}{frames/practice}

\begin{document}
%\transfade[duration=0.5]
\transdissolve[duration=0.5]
\begin{frame}
{Beanoves demonstration manual}
{Introduction}
\useMyIntroduction
\end{frame}

\begin{frame}
{Beanoves demonstration manual}
{Beamer facts}
\transpush[direction=180]
\useMyFacts
\end{frame}

%
% STEP 1
%
\begin{frame}
{Beanoves example about animals: Simple items}
{Slide \insertslidenumber}
\transpush[direction=180]
\useMyStepI
\end{frame}
%%
% STEP 2
%
\Beanoves {
  Sticky1 = 5::1,
  Sticky2 = Sticky1.next::1,
  Sticky3 = Sticky2.next::3,
  Sticky4 = Sticky3.next::3,
}
\begin{frame}
{Beanoves example: Uncovered items}
{Slide \insertslidenumber}
\transpush[direction=180]
\useMyStepII
\end{frame}
%
% Step 2 chronology
%
\Beanoves {
  Sticky1 = 1::7,
}
\begin{frame}
{Beanoves example: Uncovered items}
{\large Chronology}
%\transpush[direction=180]
\useMyChronologyII
\end{frame}
%
%% Step 3
%
\Beanoves {
  Air      = 1 : Gannet.last,
  Chameleo = Air.2::1,
  Gannet   = Chameleo.next::2,
  Water    = Air.next : Picasso.last,
  Octopus  = Water.2::1,
  Starfish = Octopus.next::2,
  StickyStarfish = Starfish.2::1,
  Picasso  = Starfish.next::1,
  PicassoTrans = 1 : Picasso.previous,
  Summary  = Water.next::1,
}
\begin{frame}
{Beanoves example: Uncovered items + images}
{Slide \insertslidenumber}
%\transpush[direction=180]
\useMyStepIII
\end{frame}

%
% Step 3 chronology
%
\Beanoves {
  Sticky1 = 1 :: 12,
  Sticky2 = Sticky1.next :: 5,
  Code = Sticky2.next : CodeSummary.last,
  CodeAir = Code.2::1,
  CodeWater = CodeAir.next::1,
  CodeSummary = CodeWater.next::1,
  Usage = Code.next::2,
}
\begin{frame}
{Beanoves example: Uncovered items + images}
{\large Chronology}
\transpush[direction=180]
\useMyChronologyIII
\end{frame}

\Beanoves {
  Air      = 1::0,
  Chameleo = 1,
  Gannet   = 1,
  Water   = 1::2,
  Octopus = 1,
  Starfish = 2,
  StickyStarfish = 1::0,
  Picasso = 2::0,
  PicassoTrans = 1 :: 2,
  Summary  = 1::0,
}
\begin{frame}
{Beanoves example: Uncovered items + images}
{\large Step back}
\transpush[direction=270]
\only<1-2> {
  \useMyStepIII
}
\only<2> {
  \tikz [
    remember picture,
    overlay,
  ] {
    \node [yshift=1\baselineskip,anchor=north] at (current page.center) [text width=0.75\textwidth] {%
\begin{myCodeBox}{l}
\string\visible < \textcolor{MyGreen}{\bfseries?(Octopus.1)} - > \{\\
\phantom{xx}\backslash only < \textcolor{MyGreen}{\bfseries?(Starfish.1)} - > \{\backslash transparent\{0.3\}\}\\
\phantom{xx}\% code to display the Octopus...\\
\}
\end{myCodeBox}
    }%
  }
}
\Sticky<1>[anchor=south,yshift=1\baselineskip](current page.south){\bfseries\begin{minipage}{0.8\textwidth}%
\begin{itemize}
\item [\myBulb] On next slide, the "Octopus" image is expected to be transparent\\
\item [\myBulb] Let us see the source code
\end{itemize}
\end{minipage}
}%
\Sticky<2>[anchor=south,yshift=1\baselineskip](current page.south){\bfseries\begin{minipage}{0.7\textwidth}\begin{itemize}%
\item[\myGear] \textcolor{MyGreen}{\texttt{\bfseries?(Octopus.1)}} is replaced by \textcolor{MyGreen}{\bfseries\texttt{5}}
and\\\textcolor{MyGreen}{\texttt{\bfseries?(Starfish.1)}} by \textcolor{MyGreen}{\bfseries\texttt{6}}\\
\item[\myGear] The octopus is displayed from slide 5 and transparent from slide 6
\end{itemize}
\end{minipage}
}%
\end{frame}
%%%
%%%
%%%
\Beanoves {
  Command = 1 :: 2,
  Example = Command.next :: 4,
}
\begin{frame}
{Beanoves manual}
{\large Defining named overlay sets}
\transpush[direction=180]
\begin{myCodeBox}{l}%
\string\Beanoves \{\\
\phantom{xx}%
\alt<?(Example.range)>{%
  {\only<?(Example.1)>{\color{MyGreen}}Air \phantom{xx}= 1\phantom{xxxxxxxx}:: 2,}
\only<?(Example.last)>{\bfseries\color{MyRed}\smash{\raisebox{-1ex}{\huge☚}}}
}{%
  \alt<?(Example.next)>{%
    \color{MyGreen}Air \phantom{xx}= 1\phantom{xxxxxxxx}:: \only<?(Example.next)>{{\bfseries\color{MyRed}3},}
  }{
    \myMeta{name\myCodek} = \myMeta{start\myCodek} :: \myMeta{length\myCodek},
  }
}\\
\alt<?(Example.range, Example.next)>{%
\phantom{xx}%
{\only<?(Example.2)->{\color{MyGreen}}Water = Air.next :: 3,}
}{%
\phantom{xx}...
}\\
\}
\end{myCodeBox}
Range starts and lengths are arithmetical expression involving raw integers as well as next \emph{\bfseries named overlay references}.\\
\vspace{0.25\baselineskip}
\begin{myLongCode}{l>{\quad$⟷$\quad}l}
\hline
\textnormal{Reference} & \textnormal{Integer value}\\\hline
\only<?(Example.1)>{\color{MyGreen}Air.1}%
\only<?(Example.2::4)>{\color{MyGreen}Water.1}%
\only<?(Command.range)>{\myMeta{name\myCodek}.1}
&
\only<?(Example.1)>{ \color{MyGreen}1}%
\only<?(Example.2::3)>{ \color{MyGreen}3}%
\only<?(Example.5)->{ \bfseries\color{MyRed}4}%
\only<?(Command.range)>{\myMeta{start\myCodek}}
\\
\only<?(Example.1)>{\color{MyGreen}Air.2}%
\only<?(Example.2::4)>{\color{MyGreen}Water.2}%
\only<?(Command.range)>{\myMeta{name\myCodek}.2}
&
\only<?(Example.1)>{ \color{MyGreen}2}%
\only<?(Example.2::3)>{ \color{MyGreen}4}%
\only<?(Example.5)->{ \bfseries\color{MyRed}5}%
\only<?(Command.range)>{\myMeta{start\myCodek}+ 1}
\\
\only<?(Example.1)>{{\color{MyGreen}Air.}\myMeta{i}}%
\only<?(Example.2::4)>{{\color{MyGreen}Water.}\myMeta{i}}%
\only<?(Command.range)>{\myMeta{name\myCodek}.\myMeta{i}}
&
\only<?(Example.1)>{\myMeta{i}}%
\only<?(Example.2::3)>{\myMeta{i}+ \color{MyGreen}2}%
\only<?(Example.5)->{\myMeta{i}+ \bfseries\color{MyRed}3}%
\only<?(Command.range)>{\myMeta{start\myCodek}+\myMeta{i}- 1}
\\
\only<?(Example.1)>{\rlap{\color{MyGreen}Air.length}}%
\only<?(Example.2::4)>{\rlap{\color{MyGreen}Water.length}}%
\visible<?(Command.range)>{\myMeta{name\myCodek}.length}
&
\only<?(Example.1)>{ \color{MyGreen}2}%
\only<?(Example.2::3)>{ \color{MyGreen}3}%
\only<?(Example.5)->{ \bfseries\color{MyRed}4}%
\only<?(Command.range)>{\myMeta{length\myCodek}}
\\
\only<?(Example.1)>{\color{MyGreen}Air.next}%
\only<?(Example.2::4)>{\color{MyGreen}Water.next}%
\only<?(Command.range)>{\myMeta{name\myCodek}.next}
&
\only<?(Example.1)>{ \color{MyGreen}3}%
\only<?(Example.2::3)>{ \color{MyGreen}6}%
\only<?(Example.5)->{ \bfseries\color{MyRed}7}%
\only<?(Command.range)>{\myMeta{start\myCodek}+\myMeta{length\myCodek}}
\\
\only<?(Example.1)>{\color{MyGreen}Air.last}%
\only<?(Example.2::4)>{\color{MyGreen}Water.last}%
\only<?(Command.range)>{\myMeta{name\myCodek}.last}
&
\only<?(Example.1)>{ \rlap{\color{MyGreen}2}}%
\only<?(Example.2::3)>{ \rlap{\color{MyGreen}5}}%
\only<?(Example.5)->{ \rlap{\bfseries\color{MyRed}6}}%
\visible<?(Command.range)>{\myMeta{start\myCodek}+\myMeta{length\myCodek}- 1}%
\\
\only<?(Example.1)>{\color{MyGreen}Air.previous}%
\only<?(Example.2::4)>{\color{MyGreen}Water.previous}%
\only<?(Command.range)>{\myMeta{name\myCodek}.previous}
&
\only<?(Example.1)>{ \color{MyGreen}0}%
\only<?(Example.2::3)>{ \color{MyGreen}2}%
\only<?(Example.5)->{ \bfseries\color{MyRed}3}%
\only<?(Command.range)>{\myMeta{start\myCodek}-1}
\\\hline
\end{myLongCode}
\vspace{1\baselineskip}
\Sticky<?(Example.0)-?(Example.last-1)>[
  anchor=south east,
  yshift=3\baselineskip,
  xshift=-1.5\baselineskip,
](current page.south east){\bfseries%
\begin{minipage}{0.45\textwidth}
Revisit the example...\\
\visible<?(Example.last-1)>{%
\myInlineCode*{Water.1 = Air.next}\\
\myInlineCode*{Air.last = Water.0}\\
}
\end{minipage}
}%
\Sticky<?(Example.last)->[
  anchor=south east,
  yshift=3\baselineskip,
  xshift=-1.5\baselineskip
](current page.south east){\bfseries%
\begin{minipage}{0.45\textwidth}
If the duration of the Air section happens to change,\\
\visible<?(Example.next)>{%
all the integer value automatically update
}
\end{minipage}
}%
\end{frame}

\Beanoves {
  Normal = 1 :: 1,
  Extended = Normal.next :: 2,
}
\begin{frame}
{Beanoves manual}
{\large Overlay specification query}
\begin{itemize}
\item Simple specifications\hfill
\myShortCodeBox[baseline=4mm]{l}{%
\backslash only < 4 > \phantom{xxxxxxxxx}\{...\}\\
\backslash only < 1 - 3 >\phantom{xxxxxx}\{...\}\\
}
\item Incremental specifications\hfill
\myShortCodeBox[baseline=4mm]{l}{%
\backslash only < + > \phantom{xxxxxxxxx}\{...\}\\
\backslash only < +(\textit{\bfseries<i>}) > \phantom{xxxx}\{...\}\\
}
\item \only<-?(Extended.0)>{\transparent{0.3}}\bfseries\color{MyGreen}Specification queries\hfill
\visible<?(Extended.range)>{%
\myShortCodeBox[baseline=2.5mm]{l}{%
\backslash only < {\color{MyGreen}?(\textit{\bfseries<query>})} > \{...\}\\
}\vphantom{\myShortCodeBox[baseline=4mm]{l}{%
x\\
x\\
}}%
}%
\end{itemize}
\vspace{1\baselineskip}
\visible<?(Extended.2)->{%
A query may be used in an overlay specification wherever an integer or a range can be.%
\\\myInlineCode{\backslash only} may be replaced by any specification aware command.
\vspace{1\baselineskip}

}%
\end{frame}
%
\Beanoves {
  Main = 1 :: 2,
  Sticky = Main.next
}
\begin{frame}
{Beanoves manual}
{\large Overlay specification query syntax}
\begin{itemize}
\item Position specifications
\begin{myCodeBox}{l}%
\only<?(Sticky)>{\color{MyGreen}\bfseries}%
?(\myMeta{integer expression with aliases})
\end{myCodeBox}
\item Explicit range specifications
\begin{myCodeBox}{l}%
\only<?(Sticky)>{\color{MyGreen}\bfseries}%
?(\myMeta{start expression} :: \myMeta{length expression})
\end{myCodeBox}
Both integer expressions accept aliases.
\item Logical range specifications with a \emph{\textbf{range alias}:}
\setlength{\tabcolsep}{0mm}
\\[0.25\baselineskip]
\begin{tabular}{>{\ttfamily}l>{\ttfamily\quad$⟷$\quad}l}
\myMeta{name\myCodek}.range
&
\myMeta{name\myCodek}.1 - \myMeta{name\myCodek}.last
\end{tabular}
\\[0.25\baselineskip]
where ``\texttt{-}'' stands for a dash and not a minus sign.
\begin{myCodeBox}{l}%
?(\myMeta{name\myCodek}.range)
\end{myCodeBox}
\end{itemize}
\visible<2->{%
Range queries and beamer ranges must not be combined like in \myInlineCode{?(Air.range)-10}, leading to the incorrect syntax \myInlineCode{1-2-10}.
}
\Sticky<?(Sticky)>[anchor=south,yshift=1.5\baselineskip](current page.south){\bfseries%
\begin{minipage}{0.75\textwidth}
\myBulb\ The middle slide of the Air topic is
\begin{center}\myInlineCode*{?((Air.1+Air.last)/2)}.\end{center}
\myBulb\ What corresponds to next query?
\begin{center}\myInlineCode*{?(Water.0 :: Water.length + 2)}\end{center}
\end{minipage}
}%
\end{frame}
%
\Beanoves {
  Main = 1 :: 4,
  Example = 0::0,
  Sticky = Main.next,
}
\begin{frame}
{Beanoves manual}
{\large Incremental specifications}
\begin{myCodeBox}{l}%
\backslash begin \{frame\}\\
\backslash Beanoves \{\\
\phantom{xx}%
\alt<?(Example.range)>{%
  {\only<?(Example.1)>{\color{MyGreen}}Air \phantom{xx}= 1\phantom{xxxxxxxx}: 2,}
}{%
  \myMeta{name\myCodek} = \myMeta{start\myCodek} :: \myMeta{length\myCodek},
}\\
\alt<?(Example.range)>{%
\phantom{xx}%
{\only<?(Example.2)->{\color{MyGreen}}Water = Air.next :: 3,}
}{%
\phantom{xx}...
}\\
\}
\end{myCodeBox}
Each logical overlay range has a \myEmph{counter}, with dedicated alias and operations.
Within a specification query:
\begin{itemize}
\item
\only<-?(Main++)>{\transparent{0.3}}%
\myInlineCode*{\myMeta{name\myCodek}}, with no trailing ``\myInlineCode{.}'', is an alias for the \myEmph{counter}
\item
\only<-?(Main++)>{\transparent{0.3}}%
\myInlineCode*{++\myMeta{name\myCodek}} stands for the \myEmph{counter} once incremented by 1
\item
\only<-?(Main++)>{\transparent{0.3}}%
\myInlineCode*{\myMeta{name\myCodek}+=\myMeta{i}} stands for the \myEmph{counter} after being incremented by \myMeta{i}.
\item
\only<-?(Main++)>{\transparent{0.3}}%
\myInlineCode*{\myMeta{name\myCodek}.reset} stands for the \myEmph{counter} once reset to its initial value.
\end{itemize}
%
\Sticky<?(++Main)>[anchor=south,yshift=1.5\baselineskip](current page.south){\bfseries%
\begin{minipage}{0.5\textwidth}
\myWatch\ Revisit the example...
\end{minipage}
}%
\vspace{2\baselineskip}
%
\end{frame}
\Beanoves {
  X1 = 1 : Sticky1b.last,
  Sticky1a = X1.1::2,
  Sticky1b = Sticky1a.next + 1 :: 2,
  W1 = Sticky1a.next :: 3,
  X2 = X1.next : Sticky2b.last,
  Sticky2a = X2.1 :: 2,
  Sticky2b = Sticky2a.next :: 3,
  W2 = Sticky2a.next :: 3,
}
\begin{frame}
{Beanoves manual}
{\large Incremental specifications in practice}
\useMyPractice
\end{frame}

\begin{frame}
{Beanoves manual}
{\large Why aliases are helpful}
\Beanoves {
  Main = 1 : 2,
}
\begin{itemize}
\item
As soon as one leaves basic frame layouts to make presentations more attractive and efficient, then \myPkg{beanoves} aliases should come into play.
\item
One can organize the slides with logical names for a better understanding: aliases and integer expressions rather than raw integers make specifications more explicit
\item
Adding or removing a slide from one slide range does not significantly affect the other slide ranges.
\end{itemize}
\end{frame}

\end{document}
