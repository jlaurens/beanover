% !TeX encoding = UTF-8
% !TeX program = lualatex
% !TeX root = ../demo.tex
\setlength{\tabcolsep}{0mm}
\renewcommand\arraystretch{0}%
\begin{tabular}{p{0.275\textwidth}p{0.725\textwidth}}
%
\only<?(X1.range)>{%
  \BealoverReset{W1}%
  \begin{itemize} [<?(++W1)->]
    \item Octopus
    \item Starfish
    \item Picasso \rlap{fish}
  \end{itemize}
}%
%
\only<?(X2.range)>{%
  \BealoverReset{W2}%
  {
  \setbeamercovered{transparent=30}
  \begin{itemize} [<?(++W2)>]
    \item Octopus
    \item Starfish
    \item Picasso \rlap{fish}
  \end{itemize}%
}%
}
&
\begin{myCodeBox}[baseline=5.5\baselineskip]{l}%
    \backslash begin \{ frame \}\\
    \backslash Bealover \{\\
     \ \ {\color{MyGreen}\bfseries Water = 1 : 3,}\\
    \}\\
    \alt<?(X1.range)>{%
      \backslash begin \{ itemize \} [ <{\color{MyGreen}\bfseries
        ?\smash{\tikz [remember picture, baseline=(O.base),inner xsep=0mm] {
          \node (O) {(++Water)};
          \only<?(Sticky1a.2)>{
            \begin{pgfinterruptboundingbox}
            \node [draw,shape=ellipse,thick,] at (O) {\phantom{ (++Water)}};
            \end{pgfinterruptboundingbox}
          }
        }}}- > ]\\
      \only<?(++W1.reset)> {\color{MyGreen}\bfseries}%
      \phantom{xx}\backslash item Octopus\\
      \only<?(++W1)>{\color{MyGreen}\bfseries}%
      \phantom{xx}\backslash item Starfish\\
      \only<?(++W1)>{\color{MyGreen}\bfseries}%
      \phantom{xx}\backslash item Picasso fish\\
    } {%
      {\only<?(Sticky2a.2)>{\color{MyGreen}\bfseries}%
      \backslash setbeamercovered \{ transparent = 30 \}}%
      \tikz[remember picture]{\coordinate (P) at (0,0);}\\
      \backslash begin \{ itemize \} [ < {\only<?(Sticky2a.2)>{\color{MyGreen}\bfseries}?(++Water)} > ]\\
      \phantom{xx}\backslash item
      {\only<?(++W2.reset)>{\bfseries}Octopus}\\
      \phantom{xx}\backslash item
      {\only<?(++W2)>{\bfseries}Starfish}\\
      \phantom{xx}\backslash item
      {\only<?(++W2)>{\bfseries}Picasso fish}\\
    }%
    \backslash end \{itemize\}\\
    \backslash end \{frame\}
  \end{myCodeBox}
\end{tabular}
\BealoverSticky<?(Sticky1a.range)>(current page.south){0.75\textwidth}[anchor=south,yshift=1.5\baselineskip]{\bfseries%
\vspace{-0.5\baselineskip}%
\begin{itemize}
\item[\myBulb]To uncover items one at a time,
\visible<?(++Sticky1a.reset+1)>{we use beamer's%
\tikz[overlay,remember picture]{
  \path[my arrow](0.1,0.1) edge [out=45, in=-45] ($(O)+0.75*(1.5,-1)$);
}
advanced overlay specifications.
}
\end{itemize}
}%
\BealoverSticky<?(Sticky1b.range)>(current page.south){0.7\textwidth}[anchor=south,yshift=1.5\baselineskip]{\bfseries%
\vspace{-0.5\baselineskip}%
\begin{itemize}
\item[\myBulb]Notice how the lines of code are hilighted\\at the same time the items are uncovered...
\end{itemize}
}%
\transpush<?(Sticky2a.1)>[direction=90]
\BealoverReset{Sticky2a}
\BealoverSticky<?(Sticky2a.range)>(current page.south){0.7\textwidth}[anchor=south,yshift=1\baselineskip]{\bfseries%
\vspace{-0.5\baselineskip}%
\begin{itemize}
\item[\myBulb]Managing transparency is straightforward
\visible<?(Sticky2a+=2)>{we use%
\tikz[overlay,remember picture]{
  \path[my arrow](0.1,0.1) edge [out=45, in=-45] ($(P)+0.3*(1,-1.4)$);
}
beamer's facilities.%
}%
\end{itemize}
}%
\BealoverSticky<?(Sticky2b.range)>(current page.south){0.7\textwidth}[anchor=south,yshift=1\baselineskip]{\bfseries%
\vspace{-0.5\baselineskip}%
\begin{itemize}
\item[\myBulb]Once again the code is properly hilighted to\\illustrate the behavior of the itemized list...
\end{itemize}
}%
%
