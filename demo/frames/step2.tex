% !TeX encoding = UTF-8
% !TeX program = lualatex
% !TeX root = ../demo.tex
\setlength{\tabcolsep}{0pt}
\renewcommand\arraystretch{0}
\begin{tabular}{p{0.35\textwidth}p{0.65\textwidth}}
\begin{itemize}
\item
Air
\begin{itemize}
\item
\only<-+>{\transparent{0.3}}
Chameleo
\item
\only<-+>{\transparent{0.3}}
Gannet
\end{itemize}
\item
\only<-+>{\transparent{0.3}}
Water
\begin{itemize}
\item
\only<-+>{\transparent{0.3}}
Octopus
\item
\only<-.>{\transparent{0.3}}
Starfish
\item
\only<-.>{\transparent{0.3}}
Picasso fish
\end{itemize}
\end{itemize}
&
\visible<?(Sticky2.last:)>{
\vspace{-2.5\baselineskip}
\begin{myCodeBox}[baseline=3\baselineskip]{l}%
\backslash begin\{itemize\}\\
\phantom{xx}\backslash item Air\\
\phantom{xx}\backslash begin\{itemize\}\\
\phantom{xxxx}\backslash item\\
\phantom{xxxx}\tikz[remember picture]\coordinate (to1) at (-0.1,0.1);%
{\only<?(Sticky3.range)>{\bfseries\color{MyGreen}}\backslash only<-+>\{\backslash transparent\{0.3\}\}}\\
\phantom{xxxx}Chameleo\\
\phantom{xxxx}\backslash item\\
\phantom{xxxx}\tikz[remember picture]\coordinate (to2) at (-0.1,0.1);%
{\only<?(Sticky3.range)>{\bfseries\color{MyGreen}}\backslash only<-+>\{\backslash transparent\{0.3\}\}}\\
\phantom{xxxx}Gannet\\
\phantom{xxxx}\backslash end\{itemize\}\\
\phantom{xx}\backslash item\\
\phantom{xx}{\tikz[remember picture]\coordinate (to3) at (-0.1,0.1);%
\only<?(Sticky3.range)>{\bfseries\color{MyGreen}}\backslash only<-+>\{\backslash transparent\{0.3\}\}}\\
\phantom{xx}Water\\
\phantom{xx}...\\
\backslash end\{itemize\}\\
\end{myCodeBox}
}
\end{tabular}
\Sticky<?(Sticky1.range)>[anchor=south,yshift=1.5\baselineskip](current page.south){\bfseries\begin{minipage}{0.66\textwidth} We jumped directly to the last slide
instead of uncovering water items one by one.\end{minipage}}
%%
\Sticky<?(Sticky2.range)>[anchor=south,yshift=1.5\baselineskip](current page.south){\bfseries\begin{minipage}{0.66\textwidth} ... and the source code reads:\end{minipage}}
%%
\visible<?(Sticky3.range)>{
\Sticky<?(Sticky3.range)>[anchor=south,yshift=\baselineskip](current page.south){\bfseries\begin{minipage}{0.9\textwidth}
\begin{itemize}
\item [\myBulb]Notice the very handy incremental overlay specifications in
\smash{\tikz[
  overlay,
  remember picture,
] {\coordinate (from) at (-0.05,0.2);}
}%
\texttt{\color{MyGreen}\backslash only<-+>\{...\}.%
}
\item<?(Sticky3.3)>[\myBulb]Unfortunately, they won't help when different layers are involved.
\end{itemize}
\end{minipage}
}}
\only<?(Sticky3.2:Sticky3.last)>{
  \tikz[
    overlay,
    remember picture,
    my/.style={
      -{Stealth},
      MyGreen,
      thick,
      densely dotted,
    },
  ] {
    \path [my] (from) edge [out=120, in=180] (to1);
    \path [my] (from) edge [out=120, in=180] (to2);
    \path [my] (from) edge [out=120, in=180] (to3);
  }%
}

%
\Sticky<?(Sticky4.range)>[anchor=south,yshift=1.5\baselineskip](current page.south){\bfseries\begin{minipage}{0.7\textwidth}
\BeanovesReset{Sticky4}
\begin{itemize}[<?(Sticky4++)->]
\item[\myBulb] Now be more appealing with images
\item[\myBulb] Logical overlay specifications will come into play
\item[\myBulb] But before, take a look at the chronology
\end{itemize}
\end{minipage}
}
