% !TeX encoding = UTF-8
% !TeX program = lualatex
% !TeX proot = ...

\documentclass{beamer}

\RequirePackage{beanoves-debug}

\begin{document}
\Beanoves {
  Main = 2,
  B = 2:4,
}
\ExplSyntaxOn
\__bnvs_DEBUG_on:
\ExplSyntaxOff
\begin{frame}
{Beanoves demo}
{\large Counters}
\BeanovesEval{B}
\only<2>{2}
Frame: \insertframenumber /
Slide: \insertslidenumber /
%\begin{myCodeBox}{l}%
%\string\Beanoves \{
%  Main = 2,
%\}
%\end{myCodeBox}
%On slide transparent
%\begin{itemize}
%\item
%\only<-?(Main++-1)>{\transparent{0.3}}%
%\myInlineCode*{\myMeta{name\myCodek}}, with no trailing ``\myInlineCode{.}'', is an alias for the \myEmph{counter}
%\item
%\only<-?(Main++-1)>{\transparent{0.3}}%
%\myInlineCode*{++\myMeta{name\myCodek}} stands for the \myEmph{counter} once incremented by 1
%\item
%\only<-?(Main++-1)>{\transparent{0.3}}%
%\myInlineCode*{\myMeta{name\myCodek}+=\myMeta{i}} stands for the \myEmph{counter} once incremented by \myMeta{i}.
%\item
%\only<-?(Main++-1)>{\transparent{0.3}}%
%\myInlineCode*{\myMeta{name\myCodek}.reset} stands for the \myEmph{counter} once reset to its initial value.
%\end{itemize}
\only<-?(Main-1)>{}
\end{frame}

\end{document}
